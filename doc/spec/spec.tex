\documentclass{book}

\usepackage{amssymb, amsmath}
\usepackage{alltt}
\usepackage{pslatex}
\usepackage{epigraph}
\usepackage{verbatim}
\usepackage{latexsym}
\usepackage{array}
\usepackage{comment}
\usepackage{makeidx}
\usepackage{listings}
\usepackage{indentfirst}
\usepackage{verbatim}
\usepackage{color}
\usepackage{url}
\usepackage{xspace}
\usepackage{hyperref}
\usepackage{stmaryrd}
\usepackage{amsmath, amsthm, amssymb}
\usepackage{graphicx}
\usepackage{euscript}
\usepackage{mathtools}
\usepackage{mathrsfs}
\usepackage{multirow,bigdelim}
\usepackage{subcaption}
\usepackage{placeins}
\usepackage{xspace}
\usepackage{ostap}
\usepackage{bm}

\makeatletter

\makeatother

\definecolor{shadecolor}{gray}{1.00}
\definecolor{darkgray}{gray}{0.30}

\def\transarrow{\xrightarrow}
\newcommand{\setarrow}[1]{\def\transarrow{#1}}

\def\padding{\phantom{X}}
\newcommand{\setpadding}[1]{\def\padding{#1}}

\def\subarrow{}
\newcommand{\setsubarrow}[1]{\def\subarrow{#1}}

\newcommand{\trule}[2]{\frac{#1}{#2}}
\newcommand{\crule}[3]{\frac{#1}{#2},\;{#3}}
\newcommand{\withenv}[2]{{#1}\vdash{#2}}
\newcommand{\trans}[3]{{#1}\transarrow{\padding{\textstyle #2}\padding}\subarrow{#3}}
\newcommand{\ctrans}[4]{{#1}\transarrow{\padding#2\padding}\subarrow{#3},\;{#4}}
\newcommand{\llang}[1]{\mbox{\lstinline[mathescape]|#1|}}
\newcommand{\pair}[2]{\inbr{{#1}\mid{#2}}}
\newcommand{\inbr}[1]{\left<{#1}\right>}
\newcommand{\highlight}[1]{\color{red}{#1}}
\newcommand{\ruleno}[1]{\eqno[\scriptsize\textsc{#1}]}
\newcommand{\rulename}[1]{\textsc{#1}}
\newcommand{\inmath}[1]{\mbox{$#1$}}
\newcommand{\lfp}[1]{fix_{#1}}
\newcommand{\gfp}[1]{Fix_{#1}}
\newcommand{\vsep}{\vspace{-2mm}}
\newcommand{\supp}[1]{\scriptsize{#1}}
\newcommand{\sembr}[1]{\llbracket{#1}\rrbracket}
\newcommand{\cd}[1]{\texttt{#1}}
\newcommand{\free}[1]{\boxed{#1}}
\newcommand{\binds}{\;\mapsto\;}
\newcommand{\dbi}[1]{\mbox{\bf{#1}}}
\newcommand{\sv}[1]{\mbox{\textbf{#1}}}
\newcommand{\bnd}[2]{{#1}\mkern-9mu\binds\mkern-9mu{#2}}
\newtheorem{lemma}{Lemma}
\newtheorem{theorem}{Theorem}
\newcommand{\meta}[1]{{\mathcal{#1}}}
\renewcommand{\emptyset}{\varnothing}
\newcommand{\dom}[1]{\mathtt{dom}\;{#1}}
\newcommand{\primi}[2]{\mathbf{#1}\;{#2}}
\newcommand{\lama}{$\lambda\mbox{\textsc{Algol}}$\xspace}
%\newcommand{\sial}{S\textit{\lower -.5ex\hbox{I}\kern -.1667em\lower .5ex\hbox {A}}\kern -.125emL\@\xspace}
\definecolor{light-gray}{gray}{0.90}
\newcommand{\graybox}[1]{\colorbox{light-gray}{#1}}

\newcommand{\defterm}[1]{\textit{#1}}
\newcommand{\nonterm}[1]{\textit{#1}}
\newcommand{\term}[1]{\graybox{#1}}
\newcommand{\token}[1]{\textsc{#1}}
\newcommand{\alt}{\s\mid\s}
\newcommand{\s}{\:\:}

\lstdefinelanguage{alm}{
keywords={skip,if,then,else,elif,fi,while,do,od,repeat,until,for,fun,local,public,return,import,length,
string,case,of,esac,when,boxed,unboxed,string,sexp,array,infix,infixl,infixr,at,before,after,true,false},
sensitive=true,
basicstyle=\small,
%commentstyle=\scriptsize\rmfamily,
keywordstyle=\ttfamily\bfseries,
identifierstyle=\ttfamily,
basewidth={0.5em,0.5em},
columns=fixed,
fontadjust=true,
literate={->}{{$\to$}}3,
morecomment=[s][\ttfamily]{(*}{*)},
morecomment=[l][\ttfamily]{--}
}

\lstset{
mathescape=true,
basicstyle=\small,
identifierstyle=\ttfamily,
keywordstyle=\bfseries,
commentstyle=\scriptsize\rmfamily,
basewidth={0.5em,0.5em},
fontadjust=true,
escapechar=!,
language=alm
}

\sloppy

\title{\lama Language Definition}

\author{Dmitry Boulytchev}

\begin{document}

\maketitle

\chapter{Introduction}

\section{General Characteristic of the Language}

\begin{itemize}
\item procedural with first-class functions~--- functions can be passed as arguments, placed in data structures,
  returned and constructed at runtime via closures mechanism;
\item with lexical static scoping;
\item strict~--- all arguments of function application are evaluated before function's body;
\item imperative~--- variables can be re-assigned, function calls can have side effects;
\item untyped~--- no static type checking is performed;
\item supports S-expressions and pattern-matching;
\item supports user-defined infix operators, including those defined in local scopes;
\item with automatic memory management (garbage collection).
\end{itemize}

\section{Notation}

Pairs and tuples:

\[
\inbr{\bullet,\,\bullet,\,\dots}
\]

Lists of elements of kind $X$:

\[
X^*
\]

Deconstructing lists into sublists:

\[
h\circ t
\]

This applies also to lists of length 1. Empty list is denoted

\[
  \epsilon
\]


For a mapping $f : X\to Y$ we use the following definition:

\[
f [x\gets y] = \lambda\,z\,.\,
\left\{
\begin{array}{rcl}
  y    &,& x = z \\
  f\;x &,& x\neq z
\end{array}
\right.
\]

Empty mapping (undefined everywhere) is denoted $\Lambda$, the domain of a mapping $f$~--- $\dom{f}$, and we abbreviate

\[
  \Lambda[x_1\gets y_1][x_2\gets y_2]\dots[x_k\gets y_k]
\]

as

\[
  [x_1\gets y_1,\,x_2\gets y_2,\,\dots,\,x_k\gets y_k]
\]

\section{Names, Values and States}

\begin{table}[t]
  \begin{tabular}{cccl}
    denotation         & instances                       & definition                                 & comments \\
    \hline
    $\mathscr X$       & $x,\,y,\,z,\,\dots$             &                                            & variables \\
    $\mathscr T$       & $\llang{C},\,\llang{D},\,\dots$ &                                            & tags (constructors) \\
    $\Sigma$           & $\sigma$                        & $\mathscr X\to\mathscr V$                  & bindings (a partial map from variables to values) \\
    $\Sigma_{\mathscr X}$ & $\inbr{\sigma,\,S}$             & $2^{\mathscr X}\times\Sigma$                 & local scope (a set of variable and bindings) \\
    $St$               & $\inbr{\sigma_g,\,ss}$          & $\Sigma\times\Sigma^*_{\mathscr X}$           & state (global bindings and a stack of local scopes) \\
    $\mathscr L$       & $l$                             &                                            & locations \\
    $M$                & $\mu$                           & $\mathscr L\to\mathscr C$                  & abstract memory (a partial map from locations to composite values) \\
    $\mathscr V$       & $v$                             & $\mathbb Z\uplus \mathscr L$               & values (integer values or locations) \\
    $\mathscr C$       &                                 & $Arr\uplus Sexp \uplus Clo$                & composite values (arrays, S-expressions or closures) \\
    $Arr$              &                                 & $\mathbb N\times (\mathbb N\to\mathscr V)$ & arrays (length and element function) \\
    $Sexp$             &                                 & $\mathscr T \times Arr$                    & S-expressions (tag and array of subvalues) \\
    $Clo$              &                                 & $\mathscr X \times \Sigma^*_{\mathscr X}$     & closures (function name and a stack of local scopes) 
  \end{tabular}
\end{table}


\chapter{Abstract Syntax and Semantics}

\chapter{Concrete Syntax}

In this chapter we describe the concrete syntax of the language as it is recognized by the parser. In the
syntactic description we will use extended Backus-Naur form with the following conventions:

\begin{itemize}
\item nonterminals are presented in \nonterm{italics};
\item concrete terminals are \term{grayed out};
\item classes of terminals are \token{CAPITALIZED};
\item a postfix ``$^\star$'' designates zero-or-more repetitions;
\item square brackets ``$[\dots]$'' designate zero-or-one repetition;
\item round brackets ``$(\dots)$'' are used for grouping;
\item alteration is denoted by ``$\alt$'', sequencing by juxaposition;
\item a colon ``$:$'' separates a nonterminal being defined from its definition.
\end{itemize}

In the description below we will take an in-line code samples in blockquotes "..." which are not considered as a
part of concrete syntax.

% !TEX TS-program = pdflatex
% !TeX spellcheck = en_US
% !TEX root = lama-spec.tex

\section{Lexical Structure}
\label{sec:lexical_structure}

The character set for the language is \textsc{ASCII}, case-sensitive. In the following lexical description we will use
the POSIX-Extended Regular Expressions in lexical definitions.

\subsection{Whitespaces and Comments}

Whitespaces and comments are \textsc{ASCII} sequences which serve as delimiters for other tokens but otherwise are
ignored.

The following characters are treated as whitespaces:

\begin{itemize}
\item blank character "\texttt{ }";
\item newline character "\texttt{\textbackslash n}";
\item carriage return character "\texttt{\textbackslash r}";
\item tabulation character "\texttt{\textbackslash t}".
\end{itemize}

Additionally, two kinds of comments are recognized:

\begin{itemize}
\item the end-of-line comment "\texttt{--}" escapes the rest of the line, including itself;
\item the block comment "\texttt{(*} ... \texttt{*)}" escapes all the text between
  "\texttt{(*}" and "\texttt{*)}".
\end{itemize}

There is a number of specific cases which have to be considered explicitly.

First, block comments can be properly nested. Then, the occurrences of comment symbols inside string literals (see below) are not
considered as comments.

End-of-line comment encountered \emph{outside} of a block comment escapes block comment symbols:

\begin{lstlisting}
    -- the following symbols are not considered as a block comment: (*
    -- same here: *)
\end{lstlisting}

Similarly, an end-of-line comment encountered inside a block comment is escaped:

\begin{lstlisting}
    (* Block comment starts here ...
       -- and ends here: *)
\end{lstlisting}

\subsection{Identifiers and Constants}

The language distinguishes identifiers, signed decimal literals, string and character literals (see Fig.~\ref{idents_and_consts}). There are
two kinds of identifiers: those beginning with uppercase characters (\token{UIDENT}) and lowercase characters (\token{LIDENT}).

String literals cannot span multiple lines; a blockquote character (") inside a string literal has to be doubled to prevent from
being considered as this literal's delimiter.

Character literals as a rule are comprised of a single \textsc{ASCII} character; if this character is a quote (') it has to be doubled. Additionally
two-character abbreviations "\textbackslash t" and "\textbackslash n" are recognized and converted into a single-character representation.

\begin{figure}[t]
  \[
  \begin{array}{rcl}
    \token{UIDENT} & = &\mbox{\texttt{[A-Z][a-zA-Z\_0-9]*}}\\
    \token{LIDENT} & = &\mbox{\texttt{[a-z][a-zA-Z\_0-9]*}}\\
    \token{DECIMAL}& = &\mbox{\texttt{-?[0-9]+}}\\
    \token{STRING} & = &\mbox{\texttt{"([\^{}\textbackslash"]|"")*"}}\\
    \token{CHAR}   & = &\mbox{\texttt{'([\^{}']|''|\textbackslash n|\textbackslash t)'}}
  \end{array}
  \]
  \caption{Identifiers and constants}
  \label{idents_and_consts}
\end{figure}


\subsection{Keywords}

The following identifiers are reserved for keywords:

\begin{lstlisting}
    after    array    at      before   box   case     do     elif     else
    esac     eta      false   fi       for   fun      if     import   infix
    infixl   infixr   lazy    od       of    public   sexp   skip     str
    syntax   then     true    val      var   while    let    in
\end{lstlisting}

\subsection{Infix Operators}

Infix operators defined as follows:

\[
\token{INFIX}=\mbox{\texttt{[+*/\%\$\#@!|\&\^{}~?<>:=\textbackslash-]+}}
\]

There is a predefined set of built-in infix operators (see Fig.~\ref{builtin_infixes}); additionally
an end-user can define custom infix operators (see Section~\ref{sec:custom_infix}). Note, sometimes 
additional whitespaces are required to disambiguate infix operator applications. For example, if a
custom infix operator "\lstinline|+-|" is defined, then the expression "\lstinline|a +- b|" can no longer be
recognized as "\lstinline|a +(-b)|". Note also that a custom operator containing "\lstinline|--|" can not be
defined due to lexical conventions.

\subsection{Delimiters}

The following symbols are treated as delimiters:

\begin{lstlisting}
    .       ,         (        )        {        }
    ;       #         ->       |
\end{lstlisting}

Note, custom infix operators can coincide with delimiters "\lstinline|#|", "\lstinline!|!", and "\lstinline|->|", which can
sometimes be misleading. 



\begin{figure}[t]
  \[
    \begin{array}{rcl}
      \defterm{compilationUnit}  & : & \nonterm{import}^\star\s\nonterm{scopeExpression}\\
      \defterm{import}           & : & \term{import}\s\token{UIDENT}\s\term{;}
    \end{array}
  \]
  \caption{Compilation unit concrete syntax}
  \label{compilation_unit}
\end{figure}

\section{Compilation Units}
\label{sec:compilation_units}

Compilation unit is a minimal structure recognized by a parser. An application can contain multiple units, compiled separatedly.
In order to use other units they have to be imported. In particular, the standard library is comprized of a number of precompiled units,
which can be imported by an end-user application.

The concrete syntax for compilation unit is shown on Fig.~\ref{compilation_unit}. Besides optional imports a unit must contain
a \nonterm{scopeExpression}, which may contain some definitions and computations. Note, a unit can not be empty. The computations described in
a unit are performed at unit initialization time (see~\ref{separate_compilation}).


\begin{figure}[t]
  \[
    \begin{array}{rcl}
      \defterm{scopeExpression}                & : & \nonterm{definition}^\star\s\nonterm{expression}\\
      \defterm{definition}                     & : & \nonterm{variableDefinition}\alt\nonterm{functionDefinition}\alt\nonterm{infixDefinition}\\
      \defterm{variableDefinition}             & : & (\s\term{local}\alt\term{public}\s)\s\nonterm{variableDefinitionSequence}\s\term{;}\\
      \defterm{variableDefinitionSequence}     & : & \nonterm{variableDefinitionSequenceItem}\s(\s\term{,}\s\nonterm{variableDefinitionSequenceItem}\s)^\star\\
      \defterm{variableDefinitionSequenceItem} & : & \token{LIDENT}\s[\s\term{=}\s\nonterm{basicExpression}\s]\\
      \defterm{functionDefinition}             & : & [\s\term{public}\s]\s\term{fun}\s\token{LIDENT}\s\term{(}\s\nonterm{functionArguments}\s\term{)}\s\nonterm{functionBody}\\
      \defterm{functionArguments}              & : & [\s\token{LIDENT}\s(\s\term{,}\s\token{LIDENT}\s)^\star\s]\\
      \defterm{functionBody}                   & : & \term{\{}\s\nonterm{scopeExpression}\s\term{\}}
    \end{array}
  \]
  \caption{Scope expression concrete syntax}
  \label{scope_expression}
\end{figure}

\section{Scope Expressions}

Scope expressions provide a mean to put expressions is a scoped context. The definitions in scoped expressions comprise of function definitions and
variable definitions (see Fig.~\ref{scope_expression}). For example:

\begin{lstlisting}
    local x, y, z; -- variable definitions

    fun id (x) {x} -- function definition
\end{lstlisting}

As scope expressions are expressions, they can be nested:

\begin{lstlisting}
    local x;

    { -- nested scope begins here
      local y;
      skip
    } -- nested scope ends here
\end{lstlisting}

The definitions on the top-level of compilation unit can be tagged as ``\lstinline|public|'', in which case they are exported and become visible by
other units which import the given one. Nested scopes can not contain public definitions.

The nesting relation has the shape of a tree, and in a concrete node of the tree all definitions in all enclosing scopes are visible:

\begin{lstlisting}
    local x;

    {local y; 
      {local z;
        skip -- x, y, and z are visible here
      };
      {local t;
        skip -- x, y, and t are visible here
      };
      skip -- x and y are visible here
    };
    skip -- only x is visible here
\end{lstlisting}

Multiple definitions of the same name in the same scope are prohibited:

\begin{lstlisting}
    local x;
    fun x () {0} -- error
\end{lstlisting}

However, a definition is a nested scope can override a definition in an enclosing one:

\begin{lstlisting}
    local x;

    {
      fun x () {0} -- ok
      skip         -- here x is associated with the function
    };

    skip -- here x is asociated with the variable
\end{lstlisting}

A function can freely use all visible definitions; in particular, functions defined in the
same scope can be mutually recursive:

\begin{lstlisting}
    local x;
    fun f () {0}

    { 
      fun g () {f () + h () + y} -- ok
      fun h () {g () + x}        -- ok
      local y;
      skip
    };
    skip
\end{lstlisting}

A variable, defined in a scope, can be attributed with an expression, calcualting its initial value.
These expressions, however, are evaluated in the order of variable declaration. Thus, while
technically it is possible to have forward references in the initialization expression, their
behaviour is undefined. For example:

\begin{lstlisting}
    local x = y + 2; -- undefined, as y is not yet initialized at this point
    local y = x + 2;
    skip
\end{lstlisting}

\begin{figure}[t]
  \[
    \begin{array}{rcll}
      \defterm{expression}        & : & \nonterm{basicExpression}\s(\s\term{;}\s\nonterm{expression}\s)&\\
      \defterm{basicExpression}   & : & \nonterm{binaryExpression}&\\
      \defterm{binaryExpression}  & : & \nonterm{binaryOperand}\s\token{INFIX}\s\nonterm{binaryOperand}&\alt\\
                                  &   & \nonterm{binaryOperand}&\\
      \defterm{binaryOperand}     & : & \nonterm{binaryExpression}&\alt\\
                                  &   & [\s\term{-}\s]\s\nonterm{postfixExpression}&\\
      \defterm{postfixExpression} & : & \nonterm{primary}&\alt\\
                                  &   & \nonterm{postfixExpression}\s\term{(}\s[\s\nonterm{expression}\s(\s\term{,}\s\nonterm{expression}\s)^\star\s]\s\term{)}&\alt\\
                                  &   & \nonterm{postfixExpression}\s\term{[}\s\nonterm{expression}\s\term{]}&\alt\\
                                  &   & \nonterm{postfixExpression}\s\term{.}\s\term{length}&\alt\\
                                  &   & \nonterm{postfixExpression}\s\term{.}\s\term{string}&\\      
      \defterm{primary}           & : & \token{DECIMAL}&\alt\\
                                  &   & \token{STRING}&\alt\\
                                  &   & \token{CHAR}&\alt\\
                                  &   & \token{LIDENT}&\alt\\
                                  &   & \term{true}&\alt\\
                                  &   & \term{false}&\alt\\
                                  &   & \term{infix}\s\token{INFIX}&\alt\\
                                  &   & \term{fun}\s\term{(}\s\nonterm{functionArguments}\s\term{)}\s\nonterm{functionBody}&\alt\\
                                  &   & \term{skip}&\alt\\
                                  &   & \term{return}\s[\s\nonterm{basicExpression}\s]&\alt\\                                 
                                  &   & \term{\{}\s\nonterm{scopeExpression}\s\term{\}}&\alt\\
                                  &   & \nonterm{listExpression}&\alt\\
                                  &   & \nonterm{arrayExpression}&\alt\\
                                  &   & \nonterm{S-expression}&\alt\\
                                  &   & \nonterm{ifExpression}&\alt\\
                                  &   & \nonterm{whileExpression}&\alt\\
                                  &   & \nonterm{repeatExpression}&\alt\\
                                  &   & \nonterm{forExpression}&\alt\\
                                  &   & \nonterm{caseExpression}&\alt\\
                                  &   & \term{(}\s\nonterm{expression}\s\term{)}&
    \end{array}
  \]
  \caption{Expression concrete syntax}
  \label{expressions}
\end{figure}

\section{Expressions}
\label{sec:expressions}

\begin{figure}
  \begin{tabular}{c|l|l}
    infix operator(s) & description & associativity \\
    \hline
    \lstinline|:=|                                                                                & assignment                         & right-associative \\
    \lstinline|:|                                                                                 & list constructor                   & right-associative \\
    \lstinline|!!|                                                                                & disjunction                        & left-associative  \\
    \lstinline|&&|                                                                                & conjunction                        & left-associative  \\
    \lstinline|==|, \lstinline|!=|,  \lstinline|<=|, \lstinline|<|, \lstinline|>=|, \lstinline|>| & integer comparisons                & non-associative   \\
    \lstinline|+|, \lstinline|-|                                                                  & addition, subtraction              & left-associative  \\
    \lstinline|*|, \lstinline|/|, \lstinline|%|                                                   & multiplication, quotent, remainder & left-associative
  \end{tabular}
\caption{The precedence and associativity of built-in infix operators}
\label{builtin_infixes}
\end{figure}

\begin{figure}
  \newcommand{\Ref}[1]{\mathcal{R}\,({#1})}
  \renewcommand{\arraystretch}{4}
  \[
    \begin{array}{cc}
      \Ref{x},\,x\;\mbox{is a variable}&\dfrac{\Ref{e}}{\Ref{\lstinline|$e$ [$\dots$]|}}\\
      \dfrac{\Ref{e_i}}{\Ref{\mbox{\lstinline|if $\dots$ then $\;e_1\;$ else $\;e_2\;$ fi|}}} & \dfrac{\Ref{e_i}}{\Ref{\mbox{\lstinline|case $\dots$ of $\;\dots\;$ -> $\;e_1\;\dots\;\dots\;$ -> $\;e_k\;$ esac|}}}\\
      \multicolumn{2}{c}{\dfrac{\Ref{e}}{\Ref{\lstinline|$\dots\;$;$\;e$|}}}
    \end{array}
  \]
  \caption{Reference inference system}
  \label{reference_inference}
\end{figure}

The syntax definition for expressions is shown on Fig.~\ref{expressions}. The top-level construct is \emph{sequential composition}, expressed
using right-associative conective "\term{;}". The basic blocks of sequential composition have the form of \nonterm{binaryExpression}, which is
a composition of infix operators and operands. The description above is given in a highly ambiguous form as it does not specify explicitly the
precedence and associativity of infix operators. The precedences and associativity of predefined built-in infix operators are shown
on Fig.~\ref{builtin_infixes} with the precedence level increasing top-to-bottom.

Apart from assignment and list constructor all other built-in infix operators operate on signed integers; in conjunction and disjunction
any non-zero value is treated as truth and zero as falsity, and the result respects this convention.

The assignment operator is unique among all others in the sense that it requires its left operand to designate a \emph{reference}. This
property is syntactically ensured using an inference system shown on Fig.~\ref{reference_inference}; here $\mathcal{R}\,(e)$ designates the
property ``$e$ is a reference''. The result of assignment operator coincides with its right operand, thus

\begin{lstlisting}
    x := y := 3
\end{lstlisting}

assigns 3 to both "\lstinline|x|" and "\lstinline|y|".

\subsection{Postifix Expressions}

There are four postfix forms of expressions:

\begin{itemize}
\item function call, designated as postfix form "\lstinline|($arg_1, \dots, arg_k$)|";
\item array element selection, designated as "\lstinline|[$index$]|";
\item built-in primitive "\lstinline|.string|", returning the string representation of the value;
\item built-in primitive "\lstinline|.length|", returning the length of boxed value.
\end{itemize}

Multiple postfixes are allowed, for example

\begin{lstlisting}
    x () [3] (1, 2, 3) . string
    x . string [4]
    x . length . string
    x . string . length
\end{lstlisting}

The basic form of expression is \nonterm{primary}. The simplest form of primary is an identifier or constant. Keywords \lstinline|true| and \lstinline|false|
designate integer constants 1 and 0 respectively, character constant is implicitly converted into its ASCII code.  String constants designate arrays
of one-byte characters. Infix constants allow to reference a functional value associated with corresponding infix operator, and functional constant (\emph{lambda-expression})
designates a anonymous functional value in the form of closure.

\subsection{\lstinline|skip| and \lstinline|return| Expressions}

Expression \lstinline|skip| can be used to designate a no-value when no action is needed (for example, in the body of unit which contains only declarations).
\lstinline|return| expression can be used to immediately copmlete the execution of current function call; optional return value can be specified.

\subsection{Arrays, Lists, and S-expresions}

\begin{figure}[t]
  \[
    \begin{array}{rcl}
      \defterm{arrayExpression} & : & \term{[}\s[\s\nonterm{expression}\s(\s\term{,}\s\nonterm{expression}\s)^\star\s]\s\term{]}\\
      \defterm{listExpression}  & : & \term{\{}\s[\s\nonterm{expression}\s(\s\term{,}\s\nonterm{expression}\s)^\star\s]\s\term{\}}\\
      \defterm{S-expression}    & : & \token{UIDENT}\s[\s\term{(}\s\nonterm{expression}\s[\s(\s\term{,}\s\nonterm{expression}\s)^\star\s]\term{)}\s]
    \end{array}
  \]
  \caption{Array, list, and S-expressions concrete syntax}  
  \label{composite_expressions}
\end{figure}

There are three forms of expressions to specify composite values: arrays, lists and S-expressions (see Fig.~\ref{composite_expressions}). Note, it is impossible
to specify one-element list as "\lstinline|{$e$}|" since it is treated as scope expression. Instead, the form "\lstinline|$e$:{}|" can be used; alternatively, standard
unit "\lstinline|List|" (see~\ref{sec:standard_library_list}) defines function "\lstinline|singleton|" which serves for the same purpose.

\subsection{Conditional Expressions}

\begin{figure}[t]
  \[
    \begin{array}{rcll}
      \defterm{ifExpression}  & : & \term{if}\s\nonterm{expression}\s\term{then}\s\nonterm{scopeExpression}\s[\s\nonterm{elsePart}\s]\s\term{fi}&\\
      \defterm{elsePart}      & : & \term{elif}\s\nonterm{expression}\s\term{then}\s\nonterm{scopeExpression}\s[\s\nonterm{elsePart}\s]&\alt\\
                              &   & \term{else}\s\nonterm{scopeExpression}&
    \end{array}
  \]
  \caption{If-expression concrete syntax}
  \label{if_expression}
\end{figure}

Conditional expression branches the control depending in the value of a certain expression; the value zero is treated as falsity, nonzero as truth. The
extended form

\begin{lstlisting}
    if $\;c_1\;$ then $\;e1\;$
    elif $\;c_2\;$ then $\;e_2\;$
    ...
    else $\;e_{k+1}\;$
    fi
\end{lstlisting}

is equivalent to the nested form

\begin{lstlisting}
    if $\;c_1\;$ then $\;e1\;$
    else if $\;c_2\;$ then $\;e_2\;$
    ...
    else $\;e_{k+1}\;$
    fi
\end{lstlisting}

\subsection{Loop Expressions}

\begin{figure}[t]
  \[
    \begin{array}{rcl}
      \defterm{whileExpression}  & : & \term{while}\s\nonterm{expression}\s\term{do}\s\nonterm{scopeExpression}\s\term{od}\\
      \defterm{repeatExpression} & : & \term{repeat}\s\nonterm{scopeExpression}\s\term{until}\s\nonterm{basicExpression}\\
      \defterm{forExpression}    & : & \term{for}\s\nonterm{expression}\s\term{,}\s\nonterm{expression}\s\term{,}\s\nonterm{expression}\\
                                 &   & \term{do}\nonterm{scopeExpresssion}\s\term{od}
    \end{array}
  \]
  \caption{Loop expressions concrete syntax}
  \label{loop_expression}
\end{figure}

There are three forms of loop expressions~--- "\lstinline|while|", "\lstinline|repeat|", and "\lstinline|for|", among which "\lstinline|while|" is the
basic one (see Fig.~\ref{loop_expression}). In "\lstinline|while|" expression the evaluation of the body is repeated as long as the evaluation of condition provides
a non-zero value. The condition is evaluated before the body on each iteration of the loop, and the body is evaluated in the context of
condition evaluation results.

The construct "\lstinline|repeat $\;e\;$ until $\;c$|" is derived and operationally equivalent to

\begin{lstlisting}
    $e\;$; while $\;c\;$ == 0 do $\;e\;$ od
\end{lstlisting}

However, the top-level local declarations in the body of "\lstinline|repeat|"-loop are visible in the condition expression:

\begin{lstlisting}
    repeat local x = read () until x 
\end{lstlisting}


The construct "\lstinline|for $\;i\;$, $\;c\;$, $\;s\;$ do $\;e\;$ od|" is also derived and operationally equivalent to

\begin{lstlisting}
    $i\;$; while $\;c\;$ do $\;e\;$; $\;s\;$ od
\end{lstlisting}

However, the top-level local definitions of the the first expression ("$i$") are visible in the rest of the construct:

\begin{lstlisting}
    for local i = 0, i < 10, i := i + 1 do write (i) od
\end{lstlisting}

\subsection{Pattern Matching}

Pattern matching construct delivers a way to discriminate on a structure of a value. This structure is specified by
means of \emph{patterns} (see Fig.~\ref{pattern}). 

\begin{figure}[t]
  \[
    \begin{array}{rcll}
      \defterm{pattern}         & : & \nonterm{consPattern}\alt\nonterm{simplePattern}&\\
      \defterm{consPattern}     & : & \nonterm{simplePattern}\s\term{:}\s\nonterm{pattern}&\\
      \defterm{simplePattern}   & : & \nonterm{wildcardPattern} & \alt\\
                                &   & \nonterm{S-exprPattern} & \alt \\
                                &   & \nonterm{arrayPattern} & \alt \\
                                &   & \nonterm{listPattern} & \alt \\
                                &   & \token{LIDENT}\s[\s\term{@}\s\nonterm{pattern} \s] & \alt \\
                                &   & [\s\term{-}\s]\s\token{DECIMAL}& \alt \\
                                &   & \token{STRING} & \alt \\
                                &   & \token{CHAR} & \alt \\
                                &   & \term{true} & \alt \\
                                &   & \term{false} & \alt \\
                                &   & \term{\#}\s\term{boxed} & \alt \\
                                &   & \term{\#}\s\term{unboxed} & \alt \\
                                &   & \term{\#}\s\term{string} & \alt \\
                                &   & \term{\#}\s\term{array} & \alt \\
                                &   & \term{\#}\s\term{sexp} & \alt \\
                                &   & \term{\#}\s\term{fun} & \alt \\
                                &   & \term{(}\s\nonterm{pattern}\s\term{)} & \\
      \defterm{wildcardPattern} & : & \term{\_} &\\
      \defterm{S-exprPattern}   & : & \token{UIDENT}\s[\s\term{(}\s\nonterm{pattern}\s(\s\term{,}\s\nonterm{pattern})^\star\s\term{)}\s] &\\
      \defterm{arrayPattern}    & : & \term{[}\s[\s\nonterm{pattern}\s(\s\term{,}\s\nonterm{pattern})^\star\s]\s\term{]} &\\
      \defterm{listPattern}     & : & \term{\{}\s[\s\nonterm{pattern}\s(\s\term{,}\s\nonterm{pattern})^\star\s]\s\term{\}} &
    \end{array}
  \]
  \caption{Pattern concrete syntax}
  \label{pattern}
\end{figure}

\begin{figure}[t]
  \[
    \begin{array}{rcl}
      \defterm{caseExpression}  & : & \term{case}\s\nonterm{expression}\s\term{of}\s\nonterm{caseBranches}\s\term{esac}\\
      \defterm{caseBranches}    & : & \nonterm{caseBranch}\s[\s(\s\term{$\mid$}\s\nonterm{caseBranch}\s)^\star\s]\\
      \defterm{caseBranch}      & : & \nonterm{pattern}\s\term{$\rightarrow$}\s\nonterm{scopeExpression}
    \end{array}
  \]
  \caption{Case-expression concrete syntax}
  \label{case_expression}
\end{figure}

\subsection{Examples}
\label{sec:expression_examples}

Some other examples with comments:

\begin{tabular}{ll}
  "\lstinline|x !! y && z + 3|" & is equivalent to "\lstinline|x !! (y && (z + 3))|"\\
  "\lstinline|x == y < 4|"      & invalid \\
  "\lstinline|x [y := 8] := 6|" & is equivalent to "\lstinline|y := 8; x [8] := 6|"\\
  "\lstinline|(write (3); x) := (write (4); z)|" & is equivalent to "\lstinline|write (3); write (4); x := z|"
\end{tabular}




\begin{figure}[t]
  \[
    \begin{array}{rcll}
      \defterm{pattern}         & : & \nonterm{consPattern}\alt\nonterm{simplePattern}&\\
      \defterm{consPattern}     & : & \nonterm{simplePattern}\s\term{:}\s\nonterm{pattern}&\\
      \defterm{simplePattern}   & : & \nonterm{wildcardPattern} & \alt\\
                                &   & \nonterm{S-exprPattern} & \alt \\
                                &   & \nonterm{arrayPattern} & \alt \\
                                &   & \nonterm{listPattern} & \alt \\
                                &   & \token{LIDENT}\s[\s\term{@}\s\nonterm{pattern} \s] & \alt \\
                                &   & [\s\term{-}\s]\s\token{DECIMAL}& \alt \\
                                &   & \token{STRING} & \alt \\
                                &   & \token{CHAR} & \alt \\
                                &   & \term{true} & \alt \\
                                &   & \term{false} & \alt \\
                                &   & \term{\#}\s\term{boxed} & \alt \\
                                &   & \term{\#}\s\term{unboxed} & \alt \\
                                &   & \term{\#}\s\term{string} & \alt \\
                                &   & \term{\#}\s\term{array} & \alt \\
                                &   & \term{\#}\s\term{sexp} & \alt \\
                                &   & \term{\#}\s\term{fun} & \alt \\
                                &   & \term{(}\s\nonterm{pattern}\s\term{)} & \\
      \defterm{wildcardPattern} & : & \term{\_} &\\
      \defterm{S-exprPattern}   & : & \token{UIDENT}\s[\s\term{(}\s\nonterm{pattern}\s(\s\term{,}\s\nonterm{pattern})^\star\s\term{)}\s] &\\
      \defterm{arrayPattern}    & : & \term{[}\s[\s\nonterm{pattern}\s(\s\term{,}\s\nonterm{pattern})^\star\s]\s\term{]} &\\
      \defterm{listPattern}     & : & \term{\{}\s[\s\nonterm{pattern}\s(\s\term{,}\s\nonterm{pattern})^\star\s]\s\term{\}} &
    \end{array}
  \]
  \caption{Pattern concrete syntax}
\end{figure}

\begin{figure}[t]
  \[
    \begin{array}{rcll}
      \defterm{ifExpression}  & : & \term{if}\s\nonterm{expression}\s\term{then}\s\nonterm{scopeExpression}\s[\s\nonterm{elsePart}\s]\s\term{fi}&\\
      \defterm{elsePart}      & : & \term{elif}\s\nonterm{expression}\s\term{then}\s\nonterm{scopeExpression}\s[\s\nonterm{elsePart}\s]&\alt\\
                              &   & \term{else}\s\nonterm{scopeExpression}&
    \end{array}
  \]
  \caption{If-expression concrete syntax}
\end{figure}

\begin{figure}[t]
  \[
    \begin{array}{rcl}
      \defterm{whileExpression}  & : & \term{while}\s\nonterm{expression}\s\term{do}\s\nonterm{scopeExpression}\s\term{od}\\
      \defterm{repeatExpression} & : & \term{repeat}\s\nonterm{scopeExpression}\s\term{until}\s\nonterm{basicExpression}\\
      \defterm{forExpression}    & : & \term{for}\s\nonterm{expression}\s\term{,}\s\nonterm{expression}\s\term{,}\s\nonterm{expression}\\
                                 &   & \term{do}\nonterm{scopeExpresssion}\s\term{od}
    \end{array}
  \]
  \caption{Loop expressions concrete syntax}  
\end{figure}

\begin{figure}[t]
  \[
    \begin{array}{rcl}
      \defterm{arrayExpression} & : & \term{[}\s[\s\nonterm{expression}\s(\s\term{,}\s\nonterm{expression}\s)^\star\s]\s\term{]}\\
      \defterm{listExpression}  & : & \term{\{}\s[\s\nonterm{expression}\s(\s\term{,}\s\nonterm{expression}\s)^\star\s]\s\term{\}}\\
      \defterm{S-expression}    & : & \token{UIDENT}\s[\s\term{(}\s\nonterm{expression}\s[\s(\s\term{,}\s\nonterm{expression}\s)^\star\s]\term{)}\s]
    \end{array}
  \]
  \caption{Array, list, and S-expressions concrete syntax}  
\end{figure}


\begin{figure}[t]
  \[
    \begin{array}{rcl}
      \defterm{caseExpression}  & : & \term{case}\s\nonterm{expression}\s\term{of}\s\nonterm{caseBranches}\s\term{esac}\\
      \defterm{caseBranches}    & : & \nonterm{caseBranch}\s[\s(\s\term{$\mid$}\s\nonterm{caseBranch}\s)^\star\s]\\
      \defterm{caseBranch}      & : & \nonterm{pattern}\s\term{$\rightarrow$}\s\nonterm{scopeExpression}
    \end{array}
  \]
  \caption{Case-expression concrete syntax}
\end{figure}



\chapter{Driver Options and Separate Compilation}

\chapter{Standard Library}


\bibliographystyle{plainurl}
\bibliography{spec}

\end{document}
