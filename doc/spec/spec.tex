\documentclass{book}

\usepackage{amssymb, amsmath}
\usepackage{alltt}
\usepackage{pslatex}
\usepackage{epigraph}
\usepackage{verbatim}
\usepackage{latexsym}
\usepackage{array}
\usepackage{comment}
\usepackage{makeidx}
\usepackage{listings}
\usepackage{indentfirst}
\usepackage{verbatim}
\usepackage{color}
\usepackage{url}
\usepackage{xspace}
\usepackage{hyperref}
\usepackage{stmaryrd}
\usepackage{amsmath, amsthm, amssymb}
\usepackage{graphicx}
\usepackage{euscript}
\usepackage{mathtools}
\usepackage{mathrsfs}
\usepackage{multirow,bigdelim}
\usepackage{subcaption}
\usepackage{placeins}
\usepackage{xspace}
\usepackage{ostap}
\usepackage{bm}

\makeatletter

\makeatother

\definecolor{shadecolor}{gray}{1.00}
\definecolor{darkgray}{gray}{0.30}

\def\transarrow{\xrightarrow}
\newcommand{\setarrow}[1]{\def\transarrow{#1}}

\def\padding{\phantom{X}}
\newcommand{\setpadding}[1]{\def\padding{#1}}

\def\subarrow{}
\newcommand{\setsubarrow}[1]{\def\subarrow{#1}}

\newcommand{\trule}[2]{\frac{#1}{#2}}
\newcommand{\crule}[3]{\frac{#1}{#2},\;{#3}}
\newcommand{\withenv}[2]{{#1}\vdash{#2}}
\newcommand{\trans}[3]{{#1}\transarrow{\padding{\textstyle #2}\padding}\subarrow{#3}}
\newcommand{\ctrans}[4]{{#1}\transarrow{\padding#2\padding}\subarrow{#3},\;{#4}}
\newcommand{\llang}[1]{\mbox{\lstinline[mathescape]|#1|}}
\newcommand{\pair}[2]{\inbr{{#1}\mid{#2}}}
\newcommand{\inbr}[1]{\left<{#1}\right>}
\newcommand{\highlight}[1]{\color{red}{#1}}
\newcommand{\ruleno}[1]{\eqno[\scriptsize\textsc{#1}]}
\newcommand{\rulename}[1]{\textsc{#1}}
\newcommand{\inmath}[1]{\mbox{$#1$}}
\newcommand{\lfp}[1]{fix_{#1}}
\newcommand{\gfp}[1]{Fix_{#1}}
\newcommand{\vsep}{\vspace{-2mm}}
\newcommand{\supp}[1]{\scriptsize{#1}}
\newcommand{\sembr}[1]{\llbracket{#1}\rrbracket}
\newcommand{\cd}[1]{\texttt{#1}}
\newcommand{\free}[1]{\boxed{#1}}
\newcommand{\binds}{\;\mapsto\;}
\newcommand{\dbi}[1]{\mbox{\bf{#1}}}
\newcommand{\sv}[1]{\mbox{\textbf{#1}}}
\newcommand{\bnd}[2]{{#1}\mkern-9mu\binds\mkern-9mu{#2}}
\newtheorem{lemma}{Lemma}
\newtheorem{theorem}{Theorem}
\newcommand{\meta}[1]{{\mathcal{#1}}}
\renewcommand{\emptyset}{\varnothing}
\newcommand{\dom}[1]{\mathtt{dom}\;{#1}}
\newcommand{\primi}[2]{\mathbf{#1}\;{#2}}
\newcommand{\lama}{$\lambda\mbox{\textsc{Algol}}$\xspace}
%\newcommand{\sial}{S\textit{\lower -.5ex\hbox{I}\kern -.1667em\lower .5ex\hbox {A}}\kern -.125emL\@\xspace}
\definecolor{light-gray}{gray}{0.90}
\newcommand{\graybox}[1]{\colorbox{light-gray}{#1}}

\newcommand{\defterm}[1]{\textit{#1}}
\newcommand{\nonterm}[1]{\textit{#1}}
\newcommand{\term}[1]{\graybox{#1}}
\newcommand{\token}[1]{\textsc{#1}}
\newcommand{\alt}{\s\mid\s}
\newcommand{\s}{\:\:}

\lstdefinelanguage{alm}{
keywords={skip,if,then,else,elif,fi,while,do,od,repeat,until,for,fun,local,public,return,import,length,
string,case,of,esac,when,boxed,unboxed,string,sexp,array,infix,infixl,infixr,at,before,after,true,false},
sensitive=true,
basicstyle=\small,
%commentstyle=\scriptsize\rmfamily,
keywordstyle=\ttfamily\bfseries,
identifierstyle=\ttfamily,
basewidth={0.5em,0.5em},
columns=fixed,
fontadjust=true,
literate={->}{{$\to$}}3,
morecomment=[s][\ttfamily]{(*}{*)},
morecomment=[l][\ttfamily]{--}
}

\lstset{
mathescape=true,
basicstyle=\small,
identifierstyle=\ttfamily,
keywordstyle=\bfseries,
commentstyle=\scriptsize\rmfamily,
basewidth={0.5em,0.5em},
fontadjust=true,
escapechar=!,
language=alm
}

\sloppy

\title{\lama Language Definition}

\author{Dmitry Boulytchev}

\begin{document}

\maketitle

\tableofcontents

\chapter{Introduction}

\section{General Characteristic of the Language}

\begin{itemize}
\item procedural with first-class functions~--- functions can be passed as arguments, placed in data structures,
  returned and constructed at runtime via closures mechanism;
\item with lexical static scoping;
\item strict~--- all arguments of function application are evaluated before function's body;
\item imperative~--- variables can be re-assigned, function calls can have side effects;
\item untyped~--- no static type checking is performed;
\item supports S-expressions and pattern-matching;
\item supports user-defined infix operators, including those defined in local scopes;
\item with automatic memory management (garbage collection).
\end{itemize}

\section{Notation}

Pairs and tuples:

\[
\inbr{\bullet,\,\bullet,\,\dots}
\]

Lists of elements of kind $X$:

\[
X^*
\]

Deconstructing lists into sublists:

\[
h\circ t
\]

This applies also to lists of length 1. Empty list is denoted

\[
  \epsilon
\]


For a mapping $f : X\to Y$ we use the following definition:

\[
f [x\gets y] = \lambda\,z\,.\,
\left\{
\begin{array}{rcl}
  y    &,& x = z \\
  f\;x &,& x\neq z
\end{array}
\right.
\]

Empty mapping (undefined everywhere) is denoted $\Lambda$, the domain of a mapping $f$~--- $\dom{f}$, and we abbreviate

\[
  \Lambda[x_1\gets y_1][x_2\gets y_2]\dots[x_k\gets y_k]
\]

as

\[
  [x_1\gets y_1,\,x_2\gets y_2,\,\dots,\,x_k\gets y_k]
\]

\section{Names, Values and States}

\begin{table}[t]
  \begin{tabular}{cccl}
    denotation         & instances                       & definition                                 & comments \\
    \hline
    $\mathscr X$       & $x,\,y,\,z,\,\dots$             &                                            & variables \\
    $\mathscr T$       & $\llang{C},\,\llang{D},\,\dots$ &                                            & tags (constructors) \\
    $\Sigma$           & $\sigma$                        & $\mathscr X\to\mathscr V$                  & bindings (a partial map from variables to values) \\
    $\Sigma_{\mathscr X}$ & $\inbr{\sigma,\,S}$             & $2^{\mathscr X}\times\Sigma$                 & local scope (a set of variable and bindings) \\
    $St$               & $\inbr{\sigma_g,\,ss}$          & $\Sigma\times\Sigma^*_{\mathscr X}$           & state (global bindings and a stack of local scopes) \\
    $\mathscr L$       & $l$                             &                                            & locations \\
    $M$                & $\mu$                           & $\mathscr L\to\mathscr C$                  & abstract memory (a partial map from locations to composite values) \\
    $\mathscr V$       & $v$                             & $\mathbb Z\uplus \mathscr L$               & values (integer values or locations) \\
    $\mathscr C$       &                                 & $Arr\uplus Sexp \uplus Clo$                & composite values (arrays, S-expressions or closures) \\
    $Arr$              &                                 & $\mathbb N\times (\mathbb N\to\mathscr V)$ & arrays (length and element function) \\
    $Sexp$             &                                 & $\mathscr T \times Arr$                    & S-expressions (tag and array of subvalues) \\
    $Clo$              &                                 & $\mathscr X \times \Sigma^*_{\mathscr X}$     & closures (function name and a stack of local scopes) 
  \end{tabular}
\end{table}


\chapter{Abstract Syntax and Semantics}

\chapter{Concrete Syntax}

In this chapter we describe the concrete syntax of the language as it is recognized by the parser. In the
syntactic description we will use extended Backus-Naur form with the following conventions:

\begin{itemize}
\item nonterminals are presented in \nonterm{italics};
\item concrete terminals are \term{grayed out};
\item classes of terminals are \token{CAPITALIZED};
\item a postfix ``$^\star$'' designates zero-or-more repetitions;
\item square brackets ``$[\dots]$'' designate zero-or-one repetition;
\item round brackets ``$(\dots)$'' are used for grouping;
\item alteration is denoted by ``$\alt$'', sequencing by juxaposition;
\item a colon ``$:$'' separates a nonterminal being defined from its definition.
\end{itemize}

In the description below we will take an in-line code samples in blockquotes "..." which are not considered as a
part of concrete syntax.

% !TEX TS-program = pdflatex
% !TeX spellcheck = en_US
% !TEX root = lama-spec.tex

\section{Lexical Structure}
\label{sec:lexical_structure}

The character set for the language is \textsc{ASCII}, case-sensitive. In the following lexical description we will use
the POSIX-Extended Regular Expressions in lexical definitions.

\subsection{Whitespaces and Comments}

Whitespaces and comments are \textsc{ASCII} sequences which serve as delimiters for other tokens but otherwise are
ignored.

The following characters are treated as whitespaces:

\begin{itemize}
\item blank character "\texttt{ }";
\item newline character "\texttt{\textbackslash n}";
\item carriage return character "\texttt{\textbackslash r}";
\item tabulation character "\texttt{\textbackslash t}".
\end{itemize}

Additionally, two kinds of comments are recognized:

\begin{itemize}
\item the end-of-line comment "\texttt{--}" escapes the rest of the line, including itself;
\item the block comment "\texttt{(*} ... \texttt{*)}" escapes all the text between
  "\texttt{(*}" and "\texttt{*)}".
\end{itemize}

There is a number of specific cases which have to be considered explicitly.

First, block comments can be properly nested. Then, the occurrences of comment symbols inside string literals (see below) are not
considered as comments.

End-of-line comment encountered \emph{outside} of a block comment escapes block comment symbols:

\begin{lstlisting}
    -- the following symbols are not considered as a block comment: (*
    -- same here: *)
\end{lstlisting}

Similarly, an end-of-line comment encountered inside a block comment is escaped:

\begin{lstlisting}
    (* Block comment starts here ...
       -- and ends here: *)
\end{lstlisting}

\subsection{Identifiers and Constants}

The language distinguishes identifiers, signed decimal literals, string and character literals (see Fig.~\ref{idents_and_consts}). There are
two kinds of identifiers: those beginning with uppercase characters (\token{UIDENT}) and lowercase characters (\token{LIDENT}).

String literals cannot span multiple lines; a blockquote character (") inside a string literal has to be doubled to prevent from
being considered as this literal's delimiter.

Character literals as a rule are comprised of a single \textsc{ASCII} character; if this character is a quote (') it has to be doubled. Additionally
two-character abbreviations "\textbackslash t" and "\textbackslash n" are recognized and converted into a single-character representation.

\begin{figure}[t]
  \[
  \begin{array}{rcl}
    \token{UIDENT} & = &\mbox{\texttt{[A-Z][a-zA-Z\_0-9]*}}\\
    \token{LIDENT} & = &\mbox{\texttt{[a-z][a-zA-Z\_0-9]*}}\\
    \token{DECIMAL}& = &\mbox{\texttt{-?[0-9]+}}\\
    \token{STRING} & = &\mbox{\texttt{"([\^{}\textbackslash"]|"")*"}}\\
    \token{CHAR}   & = &\mbox{\texttt{'([\^{}']|''|\textbackslash n|\textbackslash t)'}}
  \end{array}
  \]
  \caption{Identifiers and constants}
  \label{idents_and_consts}
\end{figure}


\subsection{Keywords}

The following identifiers are reserved for keywords:

\begin{lstlisting}
    after    array    at      before   box   case     do     elif     else
    esac     eta      false   fi       for   fun      if     import   infix
    infixl   infixr   lazy    od       of    public   sexp   skip     str
    syntax   then     true    val      var   while    let    in
\end{lstlisting}

\subsection{Infix Operators}

Infix operators defined as follows:

\[
\token{INFIX}=\mbox{\texttt{[+*/\%\$\#@!|\&\^{}~?<>:=\textbackslash-]+}}
\]

There is a predefined set of built-in infix operators (see Fig.~\ref{builtin_infixes}); additionally
an end-user can define custom infix operators (see Section~\ref{sec:custom_infix}). Note, sometimes 
additional whitespaces are required to disambiguate infix operator applications. For example, if a
custom infix operator "\lstinline|+-|" is defined, then the expression "\lstinline|a +- b|" can no longer be
recognized as "\lstinline|a +(-b)|". Note also that a custom operator containing "\lstinline|--|" can not be
defined due to lexical conventions.

\subsection{Delimiters}

The following symbols are treated as delimiters:

\begin{lstlisting}
    .       ,         (        )        {        }
    ;       #         ->       |
\end{lstlisting}

Note, custom infix operators can coincide with delimiters "\lstinline|#|", "\lstinline!|!", and "\lstinline|->|", which can
sometimes be misleading. 



\begin{figure}[t]
  \[
    \begin{array}{rcl}
      \defterm{compilationUnit}  & : & \nonterm{import}^\star\s\nonterm{scopeExpression}\\
      \defterm{import}           & : & \term{import}\s\token{UIDENT}\s\term{;}
    \end{array}
  \]
  \caption{Compilation unit concrete syntax}
  \label{compilation_unit}
\end{figure}

\section{Compilation Units}
\label{sec:compilation_units}

Compilation unit is a minimal structure recognized by a parser. An application can contain multiple units, compiled separatedly.
In order to use other units they have to be imported. In particular, the standard library is comprized of a number of precompiled units,
which can be imported by an end-user application.

The concrete syntax for compilation unit is shown on Fig.~\ref{compilation_unit}. Besides optional imports a unit must contain
a \nonterm{scopeExpression}, which may contain some definitions and computations. Note, a unit can not be empty. The computations described in
a unit are performed at unit initialization time (see~\ref{separate_compilation}).


\begin{figure}[t]
  \[
    \begin{array}{rcl}
      \defterm{scopeExpression}                & : & \nonterm{definition}^\star\s\nonterm{expression}\\
      \defterm{definition}                     & : & \nonterm{variableDefinition}\alt\nonterm{functionDefinition}\alt\nonterm{infixDefinition}\\
      \defterm{variableDefinition}             & : & (\s\term{local}\alt\term{public}\s)\s\nonterm{variableDefinitionSequence}\s\term{;}\\
      \defterm{variableDefinitionSequence}     & : & \nonterm{variableDefinitionSequenceItem}\s(\s\term{,}\s\nonterm{variableDefinitionSequenceItem}\s)^\star\\
      \defterm{variableDefinitionSequenceItem} & : & \token{LIDENT}\s[\s\term{=}\s\nonterm{basicExpression}\s]\\
      \defterm{functionDefinition}             & : & [\s\term{public}\s]\s\term{fun}\s\token{LIDENT}\s\term{(}\s\nonterm{functionArguments}\s\term{)}\s\nonterm{functionBody}\\
      \defterm{functionArguments}              & : & [\s\token{LIDENT}\s(\s\term{,}\s\token{LIDENT}\s)^\star\s]\\
      \defterm{functionBody}                   & : & \term{\{}\s\nonterm{scopeExpression}\s\term{\}}
    \end{array}
  \]
  \caption{Scope expression concrete syntax}
  \label{scope_expression}
\end{figure}

\section{Scope Expressions}

Scope expressions provide a mean to put expressions is a scoped context. The definitions in scoped expressions comprise of function definitions and
variable definitions (see Fig.~\ref{scope_expression}). For example:

\begin{lstlisting}
    local x, y, z; -- variable definitions

    fun id (x) {x} -- function definition
\end{lstlisting}

As scope expressions are expressions, they can be nested:

\begin{lstlisting}
    local x;

    { -- nested scope begins here
      local y;
      skip
    } -- nested scope ends here
\end{lstlisting}

The definitions on the top-level of compilation unit can be tagged as ``\lstinline|public|'', in which case they are exported and become visible by
other units which import the given one. Nested scopes can not contain public definitions.

The nesting relation has the shape of a tree, and in a concrete node of the tree all definitions in all enclosing scopes are visible:

\begin{lstlisting}
    local x;

    {local y; 
      {local z;
        skip -- x, y, and z are visible here
      };
      {local t;
        skip -- x, y, and t are visible here
      };
      skip -- x and y are visible here
    };
    skip -- only x is visible here
\end{lstlisting}

Multiple definitions of the same name in the same scope are prohibited:

\begin{lstlisting}
    local x;
    fun x () {0} -- error
\end{lstlisting}

However, a definition is a nested scope can override a definition in an enclosing one:

\begin{lstlisting}
    local x;

    {
      fun x () {0} -- ok
      skip         -- here x is associated with the function
    };

    skip -- here x is asociated with the variable
\end{lstlisting}

A function can freely use all visible definitions; in particular, functions defined in the
same scope can be mutually recursive:

\begin{lstlisting}
    local x;
    fun f () {0}

    { 
      fun g () {f () + h () + y} -- ok
      fun h () {g () + x}        -- ok
      local y;
      skip
    };
    skip
\end{lstlisting}

A variable, defined in a scope, can be attributed with an expression, calcualting its initial value.
These expressions, however, are evaluated in the order of variable declaration. Thus, while
technically it is possible to have forward references in the initialization expression, their
behaviour is undefined. For example:

\begin{lstlisting}
    local x = y + 2; -- undefined, as y is not yet initialized at this point
    local y = x + 2;
    skip
\end{lstlisting}

\begin{figure}[t]
  \[
    \begin{array}{rcll}
      \defterm{expression}        & : & \nonterm{basicExpression}\s(\s\term{;}\s\nonterm{expression}\s)&\\
      \defterm{basicExpression}   & : & \nonterm{binaryExpression}&\\
      \defterm{binaryExpression}  & : & \nonterm{binaryOperand}\s\token{INFIX}\s\nonterm{binaryOperand}&\alt\\
                                  &   & \nonterm{binaryOperand}&\\
      \defterm{binaryOperand}     & : & \nonterm{binaryExpression}&\alt\\
                                  &   & [\s\term{-}\s]\s\nonterm{postfixExpression}&\\
      \defterm{postfixExpression} & : & \nonterm{primary}&\alt\\
                                  &   & \nonterm{postfixExpression}\s\term{(}\s[\s\nonterm{expression}\s(\s\term{,}\s\nonterm{expression}\s)^\star\s]\s\term{)}&\alt\\
                                  &   & \nonterm{postfixExpression}\s\term{[}\s\nonterm{expression}\s\term{]}&\alt\\
                                  &   & \nonterm{postfixExpression}\s\term{.}\s\term{length}&\alt\\
                                  &   & \nonterm{postfixExpression}\s\term{.}\s\term{string}&\\      
      \defterm{primary}           & : & \token{DECIMAL}&\alt\\
                                  &   & \token{STRING}&\alt\\
                                  &   & \token{CHAR}&\alt\\
                                  &   & \token{LIDENT}&\alt\\
                                  &   & \term{true}&\alt\\
                                  &   & \term{false}&\alt\\
                                  &   & \term{infix}\s\token{INFIX}&\alt\\
                                  &   & \term{fun}\s\term{(}\s\nonterm{functionArguments}\s\term{)}\s\nonterm{functionBody}&\alt\\
                                  &   & \term{skip}&\alt\\
                                  &   & \term{return}\s[\s\nonterm{basicExpression}\s]&\alt\\                                 
                                  &   & \term{\{}\s\nonterm{scopeExpression}\s\term{\}}&\alt\\
                                  &   & \nonterm{listExpression}&\alt\\
                                  &   & \nonterm{arrayExpression}&\alt\\
                                  &   & \nonterm{S-expression}&\alt\\
                                  &   & \nonterm{ifExpression}&\alt\\
                                  &   & \nonterm{whileExpression}&\alt\\
                                  &   & \nonterm{repeatExpression}&\alt\\
                                  &   & \nonterm{forExpression}&\alt\\
                                  &   & \nonterm{caseExpression}&\alt\\
                                  &   & \term{(}\s\nonterm{expression}\s\term{)}&
    \end{array}
  \]
  \caption{Expression concrete syntax}
  \label{expressions}
\end{figure}

\section{Expressions}
\label{sec:expressions}

\begin{figure}
  \begin{tabular}{c|l|l}
    infix operator(s) & description & associativity \\
    \hline
    \lstinline|:=|                                                                                & assignment                         & right-associative \\
    \lstinline|:|                                                                                 & list constructor                   & right-associative \\
    \lstinline|!!|                                                                                & disjunction                        & left-associative  \\
    \lstinline|&&|                                                                                & conjunction                        & left-associative  \\
    \lstinline|==|, \lstinline|!=|,  \lstinline|<=|, \lstinline|<|, \lstinline|>=|, \lstinline|>| & integer comparisons                & non-associative   \\
    \lstinline|+|, \lstinline|-|                                                                  & addition, subtraction              & left-associative  \\
    \lstinline|*|, \lstinline|/|, \lstinline|%|                                                   & multiplication, quotent, remainder & left-associative
  \end{tabular}
\caption{The precedence and associativity of built-in infix operators}
\label{builtin_infixes}
\end{figure}

\begin{figure}
  \newcommand{\Ref}[1]{\mathcal{R}\,({#1})}
  \renewcommand{\arraystretch}{4}
  \[
    \begin{array}{cc}
      \Ref{x},\,x\;\mbox{is a variable}&\dfrac{\Ref{e}}{\Ref{\lstinline|$e$ [$\dots$]|}}\\
      \dfrac{\Ref{e_i}}{\Ref{\mbox{\lstinline|if $\dots$ then $\;e_1\;$ else $\;e_2\;$ fi|}}} & \dfrac{\Ref{e_i}}{\Ref{\mbox{\lstinline|case $\dots$ of $\;\dots\;$ -> $\;e_1\;\dots\;\dots\;$ -> $\;e_k\;$ esac|}}}\\
      \multicolumn{2}{c}{\dfrac{\Ref{e}}{\Ref{\lstinline|$\dots\;$;$\;e$|}}}
    \end{array}
  \]
  \caption{Reference inference system}
  \label{reference_inference}
\end{figure}

The syntax definition for expressions is shown on Fig.~\ref{expressions}. The top-level construct is \emph{sequential composition}, expressed
using right-associative conective "\term{;}". The basic blocks of sequential composition have the form of \nonterm{binaryExpression}, which is
a composition of infix operators and operands. The description above is given in a highly ambiguous form as it does not specify explicitly the
precedence and associativity of infix operators. The precedences and associativity of predefined built-in infix operators are shown
on Fig.~\ref{builtin_infixes} with the precedence level increasing top-to-bottom.

Apart from assignment and list constructor all other built-in infix operators operate on signed integers; in conjunction and disjunction
any non-zero value is treated as truth and zero as falsity, and the result respects this convention.

The assignment operator is unique among all others in the sense that it requires its left operand to designate a \emph{reference}. This
property is syntactically ensured using an inference system shown on Fig.~\ref{reference_inference}; here $\mathcal{R}\,(e)$ designates the
property ``$e$ is a reference''. The result of assignment operator coincides with its right operand, thus

\begin{lstlisting}
    x := y := 3
\end{lstlisting}

assigns 3 to both "\lstinline|x|" and "\lstinline|y|".

\subsection{Postifix Expressions}

There are four postfix forms of expressions:

\begin{itemize}
\item function call, designated as postfix form "\lstinline|($arg_1, \dots, arg_k$)|";
\item array element selection, designated as "\lstinline|[$index$]|";
\item built-in primitive "\lstinline|.string|", returning the string representation of the value;
\item built-in primitive "\lstinline|.length|", returning the length of boxed value.
\end{itemize}

Multiple postfixes are allowed, for example

\begin{lstlisting}
    x () [3] (1, 2, 3) . string
    x . string [4]
    x . length . string
    x . string . length
\end{lstlisting}

The basic form of expression is \nonterm{primary}. The simplest form of primary is an identifier or constant. Keywords \lstinline|true| and \lstinline|false|
designate integer constants 1 and 0 respectively, character constant is implicitly converted into its ASCII code.  String constants designate arrays
of one-byte characters. Infix constants allow to reference a functional value associated with corresponding infix operator, and functional constant (\emph{lambda-expression})
designates a anonymous functional value in the form of closure.

\subsection{\lstinline|skip| and \lstinline|return| Expressions}

Expression \lstinline|skip| can be used to designate a no-value when no action is needed (for example, in the body of unit which contains only declarations).
\lstinline|return| expression can be used to immediately copmlete the execution of current function call; optional return value can be specified.

\subsection{Arrays, Lists, and S-expresions}

\begin{figure}[t]
  \[
    \begin{array}{rcl}
      \defterm{arrayExpression} & : & \term{[}\s[\s\nonterm{expression}\s(\s\term{,}\s\nonterm{expression}\s)^\star\s]\s\term{]}\\
      \defterm{listExpression}  & : & \term{\{}\s[\s\nonterm{expression}\s(\s\term{,}\s\nonterm{expression}\s)^\star\s]\s\term{\}}\\
      \defterm{S-expression}    & : & \token{UIDENT}\s[\s\term{(}\s\nonterm{expression}\s[\s(\s\term{,}\s\nonterm{expression}\s)^\star\s]\term{)}\s]
    \end{array}
  \]
  \caption{Array, list, and S-expressions concrete syntax}  
  \label{composite_expressions}
\end{figure}

There are three forms of expressions to specify composite values: arrays, lists and S-expressions (see Fig.~\ref{composite_expressions}). Note, it is impossible
to specify one-element list as "\lstinline|{$e$}|" since it is treated as scope expression. Instead, the form "\lstinline|$e$:{}|" can be used; alternatively, standard
unit "\lstinline|List|" (see~\ref{sec:standard_library_list}) defines function "\lstinline|singleton|" which serves for the same purpose.

\subsection{Conditional Expressions}

\begin{figure}[t]
  \[
    \begin{array}{rcll}
      \defterm{ifExpression}  & : & \term{if}\s\nonterm{expression}\s\term{then}\s\nonterm{scopeExpression}\s[\s\nonterm{elsePart}\s]\s\term{fi}&\\
      \defterm{elsePart}      & : & \term{elif}\s\nonterm{expression}\s\term{then}\s\nonterm{scopeExpression}\s[\s\nonterm{elsePart}\s]&\alt\\
                              &   & \term{else}\s\nonterm{scopeExpression}&
    \end{array}
  \]
  \caption{If-expression concrete syntax}
  \label{if_expression}
\end{figure}

Conditional expression branches the control depending in the value of a certain expression; the value zero is treated as falsity, nonzero as truth. The
extended form

\begin{lstlisting}
    if $\;c_1\;$ then $\;e1\;$
    elif $\;c_2\;$ then $\;e_2\;$
    ...
    else $\;e_{k+1}\;$
    fi
\end{lstlisting}

is equivalent to the nested form

\begin{lstlisting}
    if $\;c_1\;$ then $\;e1\;$
    else if $\;c_2\;$ then $\;e_2\;$
    ...
    else $\;e_{k+1}\;$
    fi
\end{lstlisting}

\subsection{Loop Expressions}

\begin{figure}[t]
  \[
    \begin{array}{rcl}
      \defterm{whileExpression}  & : & \term{while}\s\nonterm{expression}\s\term{do}\s\nonterm{scopeExpression}\s\term{od}\\
      \defterm{repeatExpression} & : & \term{repeat}\s\nonterm{scopeExpression}\s\term{until}\s\nonterm{basicExpression}\\
      \defterm{forExpression}    & : & \term{for}\s\nonterm{expression}\s\term{,}\s\nonterm{expression}\s\term{,}\s\nonterm{expression}\\
                                 &   & \term{do}\nonterm{scopeExpresssion}\s\term{od}
    \end{array}
  \]
  \caption{Loop expressions concrete syntax}
  \label{loop_expression}
\end{figure}

There are three forms of loop expressions~--- "\lstinline|while|", "\lstinline|repeat|", and "\lstinline|for|", among which "\lstinline|while|" is the
basic one (see Fig.~\ref{loop_expression}). In "\lstinline|while|" expression the evaluation of the body is repeated as long as the evaluation of condition provides
a non-zero value. The condition is evaluated before the body on each iteration of the loop, and the body is evaluated in the context of
condition evaluation results.

The construct "\lstinline|repeat $\;e\;$ until $\;c$|" is derived and operationally equivalent to

\begin{lstlisting}
    $e\;$; while $\;c\;$ == 0 do $\;e\;$ od
\end{lstlisting}

However, the top-level local declarations in the body of "\lstinline|repeat|"-loop are visible in the condition expression:

\begin{lstlisting}
    repeat local x = read () until x 
\end{lstlisting}


The construct "\lstinline|for $\;i\;$, $\;c\;$, $\;s\;$ do $\;e\;$ od|" is also derived and operationally equivalent to

\begin{lstlisting}
    $i\;$; while $\;c\;$ do $\;e\;$; $\;s\;$ od
\end{lstlisting}

However, the top-level local definitions of the the first expression ("$i$") are visible in the rest of the construct:

\begin{lstlisting}
    for local i = 0, i < 10, i := i + 1 do write (i) od
\end{lstlisting}

\subsection{Pattern Matching}

Pattern matching construct delivers a way to discriminate on a structure of a value. This structure is specified by
means of \emph{patterns} (see Fig.~\ref{pattern}). 

\begin{figure}[t]
  \[
    \begin{array}{rcll}
      \defterm{pattern}         & : & \nonterm{consPattern}\alt\nonterm{simplePattern}&\\
      \defterm{consPattern}     & : & \nonterm{simplePattern}\s\term{:}\s\nonterm{pattern}&\\
      \defterm{simplePattern}   & : & \nonterm{wildcardPattern} & \alt\\
                                &   & \nonterm{S-exprPattern} & \alt \\
                                &   & \nonterm{arrayPattern} & \alt \\
                                &   & \nonterm{listPattern} & \alt \\
                                &   & \token{LIDENT}\s[\s\term{@}\s\nonterm{pattern} \s] & \alt \\
                                &   & [\s\term{-}\s]\s\token{DECIMAL}& \alt \\
                                &   & \token{STRING} & \alt \\
                                &   & \token{CHAR} & \alt \\
                                &   & \term{true} & \alt \\
                                &   & \term{false} & \alt \\
                                &   & \term{\#}\s\term{boxed} & \alt \\
                                &   & \term{\#}\s\term{unboxed} & \alt \\
                                &   & \term{\#}\s\term{string} & \alt \\
                                &   & \term{\#}\s\term{array} & \alt \\
                                &   & \term{\#}\s\term{sexp} & \alt \\
                                &   & \term{\#}\s\term{fun} & \alt \\
                                &   & \term{(}\s\nonterm{pattern}\s\term{)} & \\
      \defterm{wildcardPattern} & : & \term{\_} &\\
      \defterm{S-exprPattern}   & : & \token{UIDENT}\s[\s\term{(}\s\nonterm{pattern}\s(\s\term{,}\s\nonterm{pattern})^\star\s\term{)}\s] &\\
      \defterm{arrayPattern}    & : & \term{[}\s[\s\nonterm{pattern}\s(\s\term{,}\s\nonterm{pattern})^\star\s]\s\term{]} &\\
      \defterm{listPattern}     & : & \term{\{}\s[\s\nonterm{pattern}\s(\s\term{,}\s\nonterm{pattern})^\star\s]\s\term{\}} &
    \end{array}
  \]
  \caption{Pattern concrete syntax}
  \label{pattern}
\end{figure}

\begin{figure}[t]
  \[
    \begin{array}{rcl}
      \defterm{caseExpression}  & : & \term{case}\s\nonterm{expression}\s\term{of}\s\nonterm{caseBranches}\s\term{esac}\\
      \defterm{caseBranches}    & : & \nonterm{caseBranch}\s[\s(\s\term{$\mid$}\s\nonterm{caseBranch}\s)^\star\s]\\
      \defterm{caseBranch}      & : & \nonterm{pattern}\s\term{$\rightarrow$}\s\nonterm{scopeExpression}
    \end{array}
  \]
  \caption{Case-expression concrete syntax}
  \label{case_expression}
\end{figure}

\subsection{Examples}
\label{sec:expression_examples}

Some other examples with comments:

\begin{tabular}{ll}
  "\lstinline|x !! y && z + 3|" & is equivalent to "\lstinline|x !! (y && (z + 3))|"\\
  "\lstinline|x == y < 4|"      & invalid \\
  "\lstinline|x [y := 8] := 6|" & is equivalent to "\lstinline|y := 8; x [8] := 6|"\\
  "\lstinline|(write (3); x) := (write (4); z)|" & is equivalent to "\lstinline|write (3); write (4); x := z|"
\end{tabular}




\begin{figure}[t]
  \[
    \begin{array}{rcll}
      \defterm{pattern}         & : & \nonterm{consPattern}\alt\nonterm{simplePattern}&\\
      \defterm{consPattern}     & : & \nonterm{simplePattern}\s\term{:}\s\nonterm{pattern}&\\
      \defterm{simplePattern}   & : & \nonterm{wildcardPattern} & \alt\\
                                &   & \nonterm{S-exprPattern} & \alt \\
                                &   & \nonterm{arrayPattern} & \alt \\
                                &   & \nonterm{listPattern} & \alt \\
                                &   & \token{LIDENT}\s[\s\term{@}\s\nonterm{pattern} \s] & \alt \\
                                &   & [\s\term{-}\s]\s\token{DECIMAL}& \alt \\
                                &   & \token{STRING} & \alt \\
                                &   & \token{CHAR} & \alt \\
                                &   & \term{true} & \alt \\
                                &   & \term{false} & \alt \\
                                &   & \term{\#}\s\term{boxed} & \alt \\
                                &   & \term{\#}\s\term{unboxed} & \alt \\
                                &   & \term{\#}\s\term{string} & \alt \\
                                &   & \term{\#}\s\term{array} & \alt \\
                                &   & \term{\#}\s\term{sexp} & \alt \\
                                &   & \term{\#}\s\term{fun} & \alt \\
                                &   & \term{(}\s\nonterm{pattern}\s\term{)} & \\
      \defterm{wildcardPattern} & : & \term{\_} &\\
      \defterm{S-exprPattern}   & : & \token{UIDENT}\s[\s\term{(}\s\nonterm{pattern}\s(\s\term{,}\s\nonterm{pattern})^\star\s\term{)}\s] &\\
      \defterm{arrayPattern}    & : & \term{[}\s[\s\nonterm{pattern}\s(\s\term{,}\s\nonterm{pattern})^\star\s]\s\term{]} &\\
      \defterm{listPattern}     & : & \term{\{}\s[\s\nonterm{pattern}\s(\s\term{,}\s\nonterm{pattern})^\star\s]\s\term{\}} &
    \end{array}
  \]
  \caption{Pattern concrete syntax}
\end{figure}

\begin{figure}[t]
  \[
    \begin{array}{rcll}
      \defterm{ifExpression}  & : & \term{if}\s\nonterm{expression}\s\term{then}\s\nonterm{scopeExpression}\s[\s\nonterm{elsePart}\s]\s\term{fi}&\\
      \defterm{elsePart}      & : & \term{elif}\s\nonterm{expression}\s\term{then}\s\nonterm{scopeExpression}\s[\s\nonterm{elsePart}\s]&\alt\\
                              &   & \term{else}\s\nonterm{scopeExpression}&
    \end{array}
  \]
  \caption{If-expression concrete syntax}
\end{figure}

\begin{figure}[t]
  \[
    \begin{array}{rcl}
      \defterm{whileExpression}  & : & \term{while}\s\nonterm{expression}\s\term{do}\s\nonterm{scopeExpression}\s\term{od}\\
      \defterm{repeatExpression} & : & \term{repeat}\s\nonterm{scopeExpression}\s\term{until}\s\nonterm{basicExpression}\\
      \defterm{forExpression}    & : & \term{for}\s\nonterm{expression}\s\term{,}\s\nonterm{expression}\s\term{,}\s\nonterm{expression}\\
                                 &   & \term{do}\nonterm{scopeExpresssion}\s\term{od}
    \end{array}
  \]
  \caption{Loop expressions concrete syntax}  
\end{figure}

\begin{figure}[t]
  \[
    \begin{array}{rcl}
      \defterm{arrayExpression} & : & \term{[}\s[\s\nonterm{expression}\s(\s\term{,}\s\nonterm{expression}\s)^\star\s]\s\term{]}\\
      \defterm{listExpression}  & : & \term{\{}\s[\s\nonterm{expression}\s(\s\term{,}\s\nonterm{expression}\s)^\star\s]\s\term{\}}\\
      \defterm{S-expression}    & : & \token{UIDENT}\s[\s\term{(}\s\nonterm{expression}\s[\s(\s\term{,}\s\nonterm{expression}\s)^\star\s]\term{)}\s]
    \end{array}
  \]
  \caption{Array, list, and S-expressions concrete syntax}  
\end{figure}


\begin{figure}[t]
  \[
    \begin{array}{rcl}
      \defterm{caseExpression}  & : & \term{case}\s\nonterm{expression}\s\term{of}\s\nonterm{caseBranches}\s\term{esac}\\
      \defterm{caseBranches}    & : & \nonterm{caseBranch}\s[\s(\s\term{$\mid$}\s\nonterm{caseBranch}\s)^\star\s]\\
      \defterm{caseBranch}      & : & \nonterm{pattern}\s\term{$\rightarrow$}\s\nonterm{scopeExpression}
    \end{array}
  \]
  \caption{Case-expression concrete syntax}
\end{figure}



\chapter{Extensions}
\label{sec:extensions}

There are some extensions for the core language defined in the previous chapters. These
extensions add some syntactic sugar, which makes writing programs in \lama a less
painful task.

\section{Custom Infix Operators}
\label{sec:custom_infix}

Besides the set of builtin infix operators (see Fig.~\ref{builtin_infixes}) users may define
custom infix operators. These operators may be declared at any scope level; when defined
at the top level they can be exported as well. However, there are some restrictions regarding the
redefinition of builtin infix operators:

\begin{itemize}
\item redefinitions of builtin infix operators can not be exported;
\item the assignment operator "\lstinline|:=|" can not be redefined.
\end{itemize}

The syntax for infix operator definition is shown on Fig.~\ref{custom_infix_construct}; a custom infix definition must specify exactly two arguments.
An associativity and precedence level has to be assigned to each custom infix operator. A precedence level is assigned by specifying at which
position, relative to other known infix operators, the operator being defined is inserted. Three kinds of specifications are allowed: at given level,
immediately before or immediately after. For example, "\lstinline|at +|" means that the operator is assigned exactly the same
level of precedence as "\lstinline|+|"; "\lstinline|after +|" creates a new precedence level immediately \emph{after} that for
"\lstinline|+|" (but \emph{before} that for "\lstinline|*|"), and "\lstinline|before *|" has exactly the same effect (provided
there were no insertions of precedence levels between those for "\lstinline|+|" and "\lstinline|*|").

When begin inserted at existing precedence level, an infix operator inherits the associativity from that level; hence, only "\lstinline|infix|"
keyword can be used for such definitions. When a new level is created, an associativity for this level has to be additionally specified
by using corresponding keyword ("\lstinline|infix|" for non-associative levels, "\lstinline|infixr|"~--- for levels with right
associativity, and "\lstinline|infixl|"~--- for levels with left associativity).

When public infix operators are exported, their relative precedence levels and associativity are exported as well; since not all 
custom infix definitions may be made public some levels may disappear from the export. For example, let us have the following definitions:

\begin{lstlisting}
    infixl ** before * (x, y) {...}
    public infixr *** before ** (x, y) {...}
\end{lstlisting}

Here in the top scope for the compilation unit we have two additional precedence levels: one for the "\lstinline|**|" and another for the "\lstinline|***|".
However, as  "\lstinline|**|" is not exported its precedence level will be forgotten during the import. Thus, only the precedence level for
"\lstinline|***|" will be created during the import as if is was defined at the level  "\lstinline|before *|".

Respectively, multiple imports of units with custom infix operators will modify the precedence level in the order of their import. For example,
if there are two units  "\lstinline|A|" and  "\lstinline|B|" with declarations "\lstinline|infixl ++ before +|" and "\lstinline|infixl +++ before +|"
correspondingly, then importing  "\lstinline|B|" after  "\lstinline|B|" will result in "\lstinline|++|" having a \emph{lower} precedence, then
"\lstinline|+++|".

\begin{figure}[t]
  \[
    \begin{array}{rcl}
      \defterm{infixDefinition} & : & \nonterm{infixHead}\s\term{(}\s\nonterm{functionArguments}\s\term{)}\s\nonterm{functionBody}\\
      \defterm{infixHead}       & : & [\s\term{public}\s]\s\nonterm{infixity}\s\token{INFIX}\s\nonterm{level}\\
      \defterm{infixity}        & : & \term{infix}\alt\term{infixl}\alt\term{infixr}\\
      \defterm{level}           & : & [\s\term{at}\alt\term{before}\alt\term{after}\s]\s\token{INFIX}
    \end{array}
  \]
  \caption{The Syntax for Infix Operator Definition}
  \label{custom_infix_construct}
\end{figure}

\section{Lazy Values and Eta-extension}

An expression

\begin{lstlisting}
    lazy$\;e$
\end{lstlisting}

where $e$~--- a $\nonterm{basicExpression}$~--- is converted into

\begin{lstlisting}
    makeLazy (fun () {$e$})    
\end{lstlisting}

where "\lstinline|makeLazy|"~--- a function from standard unit "\lstinline|Lazy|" (see Section~\ref{sec:std:lazy}). An import for
"\lstinline|Lazy|" is added implicitly.

An expression

\begin{lstlisting}
    eta$\;e$
\end{lstlisting}

where $e$~--- a $\nonterm{basicExpression}$~--- is converted into

\begin{lstlisting}
    fun ($x$) {$e$ ($x$)})    
\end{lstlisting}

where "$x$"~--- a fresh variable which does not occur free in "$e$".

\section{Dot Notation}

A function call

\begin{lstlisting}
    $f$ ($e_1$, ..., $e_k$)
\end{lstlisting}

where $f$~--- an identifier~--- can be rewritten as

\begin{lstlisting}
    $e_1$.$f$ ($e_2$, ..., $e_k$)
\end{lstlisting}

In particular, a call to a one-argument function $f (e)$ can be rewritten as $e.f$.

\section{Patterns in Function Arguments}

Patterns can be used in function argument specification: a declaration

\begin{lstlisting}
    fun f ($p_1$, ..., $p_k$) { e }
\end{lstlisting}

is equivalent to

\begin{lstlisting}
    fun f ($x_1$, ..., $x_k$) {
      case $x_1$ of
        $p_1$ -> case $x_2$ of
                   ... -> e
                 esac
      esac
    }
\end{lstlisting}

where $x_i$~--- fresh variables, not free in $e$.


\chapter{Driver Options and Separate Compilation}
\label{sec:driver}

Driver is a command-line unitility "\texttt{lamac}" which controls the invocation of the compiler. The
general format of invocation is

\begin{lstlisting}
    lamac $\;options\; filename$
\end{lstlisting}

Only one file name can be processed at once, the file name and the options can be specified in
arbitrary order.

The driver operates in a few modes:

\begin{itemize}
\item Interpreter mode. Performs an interpretation of a source program using the reference source-level interpreter ("\texttt{-i}") or
  compiles and runs a source on the stack machine ("\texttt{-s}"). In this mode separate compilation is not supported, thus no external
  units can be accessed (including "\lstinline|Std|"), only the standard set of builtins is available. 
\item Native mode, compilation ("\lstinline{-c}"). Compiles a source file into native code and writes an object file. All referenced
  external unit interfaces must have to be accessible; however no linking is performed and no executable is built.
\item Native mode, build (default). Same as for the native compilation, but additionally performs linking with the runtime library and
  all external units object files, generating executable.
\end{itemize}

In the native modes, the driver also creates import files ("\texttt{.i}") which are required for external units import to
work properly. These files has to reside in the same directory as object files for corresponding units.

Each natively compiled object file implicitly references all imported units; the top-level expression of each
unit is compiled into \emph{unit initialization procedure}, which calls unit initialization procedures of
all imported units in the same order these unites were imported. It is guaranteed that unit initialization
procedure for each unit will be called only once (regardless of the imports' shape for the whole application).

Additionally, the following options can be given to the driver:

\begin{itemize}
\item "\texttt{-I $path$}"~--- specifies a path to look for external units. Multiples instances of this option can be given in driver's
  invocation, and the paths are looked up in that order.
\item "\texttt{-dp}"~--- forces the driver to dump the AST of compiled unit. The dump is written in the file with the same
  basename as the source one, with the extension replaced with "\texttt{.ast}".
\item "\texttt{-ds}"~--- forces the driver to sump stack machine code. The option is only in effect in stack interpreter on
  native mode. The sump is written in the file "\texttt{.sm}".
\item "\texttt{-v}"~--- makes the dirver to print the version of the compiler.
\item "\texttt{-h}"~--- makes the driver to print the helpon options.
\end{itemize}

Apart from the paths specified by the "\texttt{-I}" option the driver uses the environment variable "\texttt{LAMA\_STD}"
to locate the runtime and standard libraries (see~\ref{sec:stdlib}). Thus, the units from standard libraries are accessible
without any "\texttt{-I}" option given.

\chapter{Standard Library}
\label{sec:stdlib}

The standard library is comprised of the runtime for the language and a set of pre-shipped units written in \lama itself.

\section{Unit \texttt{Std}}
\label{sec:std}

The unit "\lstinline|Std|" provides the interface for the runtime of the language. The implementation of
entities, defined in "\lstinline|Std|", resides in the runtime itself. The import of "\lstinline|Std|"
is added implicitly by the compiler and can not be specified by an end user.

The following declarations are accessible:

\descr{\lstinline|fun stringInt (s)|}{Converts a string representation of a signed decimal number into integer.}

\descr{\lstinline|fun read ()|}{Reads an integer value from the standard input, printing a prompt "\lstinline|>|".}

\descr{\lstinline|fun write (int)|}{Writes an integer value to the standard output.}

\descr{\lstinline|sysargs|}{A variable which holds an array of command-line arguments of the application (including the
name of the executable itself).}

\descr{\lstinline|fun makeArray (size)|}{Creates a fresh array of a given length. The elements of the array are left uninitialized.}

\descr{\lstinline|fun makeString (size)|}{Creates a fresh string of a given length. The elements of the string are left uninitialized.}

\descr{\lstinline|fun stringcat (list)|}{Takes a list of strings and returns the concatenates all its elements.}

\descr{\lstinline|fun matchSubString (subj, patt, pos)|}{Takes two strings "\lstinline|subj|" and "\lstinline|patt|" and integer position "\lstinline|pos|" and
checks if a substring of "\lstinline|subj|" starting at position "\lstinline|pos|" is equal to "\lstinline|patt|"; returns integer value, treated as a boolean.}

\descr{\lstinline|fun sprintf (fmt, ...)|}{Takes a format string (as per GNU C Library~\cite{GNUCLib}) and a variable number of arguments and
returns a string, acquired via processing these arguments according to the format string. Note: indexed arguments are not supported.}

\descr{\lstinline|fun substring (str, pos, len)|}{Takes a string, an integer position and length, and returns a substring of requested length of
  given string starting from given position. Raises an error if the original string is shorter then \lstinline|pos+len-1|.}

\descr{\lstinline|infix ++ at + (str1, str2)|}{String concatenation infix operator.}

\descr{\lstinline|fun clone (value)|}{Performs a shallow cloning of the argument value.}

\descr{\lstinline|fun hash (value)|}{Returns integer hash for the argument value; also works for cyclic data structures.}

\descr{\lstinline|fun tagHash (s)|}{Returns an integer value for a hash of tag, represented by string \lstinline|s|.}

\descr{\lstinline|fun compare (value1, value2)|}{Performs a structural deep comparison of two values. Determines a
  linear order relation for every pairs of values. Returns \lstinline|0| if the values are structurally equal, negative or
  positive integers otherwise. May not work for cyclic data structures.}

\descr{\lstinline|fun flatCompare (x, y)|}{Performs a shallow comparison of two values. The result is similar to that for \lstinline|compare|.}

\descr{\lstinline|fun fst (value)|}{Returns the first subvalue for a given boxed value.}

\descr{\lstinline|fun snd (value)|}{Returns the second subvalue for a given boxed value.}

\descr{\lstinline|fun hd (value)|}{Returns the head of a given list.}

\descr{\lstinline|fun tl (value)|}{Return the tail of a given list.}

\descr{\lstinline|fun readLine ()|}{Reads a line from the standard input and returns it as a string. Return "\lstinline|0|" if end
of standard input was encountered.}

\descr{\lstinline|fun printf (fmt, ...)|}{Takes a format string (as per GNU C Library~\cite{GNUCLib} and a variable number of arguments and
prints these arguments on the standard output, according to the format string.}

\descr{\lstinline|fun fopen (fname, mode)|}{Opens a file of given name in a given mode. Both arguments are strings, the return value is
an external pointer to file structure.}

\descr{\lstinline|fun fclose (file)|}{Closes a file. The file argument should be that acquired by "\lstinline|fopen|" function.}

\descr{\lstinline|fun fread (fname)|}{Reads a file content and returns it as a string. The argument is a file name as a string, the file
is automatically open and closed within the call.}

\descr{\lstinline|fun fwrite (fname, contents)|}{Writes a file. The arguments are file name and the contents to write as strings. The file
is automatically created and closed within the call.}

\descr{\lstinline|fun fprintf (file, fmt, ...)|}{Same as "\lstinline|printf|", but outputs to a given file. The file argument should be that acquired
  by \lstinline|fopen| function.}

\descr{\lstinline|fun regexp (str)|}{Compiles a string representation of a regular expression (as per GNULib's regexp~\cite{GNULib}) into
  an internal representation. The return value is a external pointer to the internal representation.}

\descr{\lstinline|fun regexpMatch (pattern, subj, pos)|}{Matches a string "\lstinline{subj}", starting from the position "\lstinline|pos|",
  against a pattern "\lstinline{pattern}". The pattern is an external pointer to a compiled representation, returned by the
  function "\lstinline|regexp|". The return value is the number of matched characters.}

\descr{\lstinline|fun failure (fmt, ...)|}{Takes a format string (as per GNU C Library~\cite{GNUCLib}, and a variable number of parameters,
  prints these parameters according to the format string on the standard error and exits. Note: indexed arguments are not supported.)}

\descr{\lstinline|fun system (cmd)|}{Executes a command in a shell. The argument is a string representing a command.}

\descr{\lstinline|fun getEnv (name)|}{Returns a value for an environment variable "\lstinline|name|". The argument is a string, the
return value is either "\lstinline|0|" (if not environment variable with given name is set), or a string value.}

\descr{\lstinline|fun random (n)|}{Returns a pseudo-random number in the interval $0..n-1$. The seed is auto-initialized by current time at 
program start time.}

\descr{\lstinline|fun time ()|}{Returns the elapsed time from program start in microseconds.}

\section{Unit \texttt{Data}}
\label{sec:data}

Generic data manupulation.

\descr{\lstinline|infix =?= at < (x, y)|}{A generic comparison operator similar to \lstinline|compare|, but capable of handling cyclic/shared data structures.}
\descr{\lstinline|infix === at == (x, y)|}{A generic equality operator capable of handling cyclic/shared data structures.}

\section{Unit \texttt{Timer}}
\label{sec:timer}

A simple timer.

\descr{\lstinline|fun timer ()|}{Creates a timer. Creates a zero-argument function which, being called, returns the elapsed time in microseconds since its creation.}
\descr{\lstinline|fun toSeconds (n)|}{Converts an integer value, interpreted as microseconds, into a floating-point string.}

\section{Unit \texttt{Random}}
\label{sec:random}

Random data structures generation functions.

\descr{\lstinline|fun randomInt ()|}{Generates a random representable integer value.}

\descr{\lstinline|fun randomString (len)|}{Generates a random string of printable ASCII characters of given length.}

\descr{\lstinline|fun randomArray (f, n)|}{Generates a random array of \emph{deep} size \lstinline|n|. The length of the array is chosen randomly, and \lstinline|f| is intended to be an element-generating function which takes the size of the element as an argument.}

\descr{\lstinline|fun split (n, k)|}{Splits a non-negative integer \lstinline|n| in \lstinline|k| random summands. Returns an array if length \lstinline|k|. \lstinline|k| has to be non-negative.}

\newsavebox\strubox

\begin{lrbox}{\strubox}
\begin{lstlisting}
    structure (100,
               [[2, fun ([x, y]) {Add (x, y)}],
                [2, fun ([x, y]) {Sub (x, y)}]],
               fun () {Const (randomInt ())}
              )    
\end{lstlisting}  
\end{lrbox}

\descr{\lstinline|fun structure (n, nodeSpec, leaf)|}{Generates a random tree-shaped data structure of size \lstinline|n|. \lstinline|nodeSpec| is an array of pairs \lstinline|[$k$, $f_k$]|, where $k$ is a non-negative integer and $f_k$ is a function which takes an array of length $k$ as its argument. Each pair describes a generator of a certain kind of interior node with degree $k$. \lstinline|leaf| is a zero-argument function which generates the leaves of the tree. For example, the following code

\usebox\strubox

can be used to generate a random arithmetic expression of size 100.}

\section{Unit \texttt{Array}}
\label{sec:array}

Array processing functions:

\descr{\lstinline|fun initArray (n, f)|}{Takes an integer value "\lstinline|n|" and a function "\lstinline|f|" and creates an array
  \[
     \mbox{\lstinline|[f (0), f (1), ..., f (n-1)]|}
  \]
}

\descr{\lstinline|fun mapArray (f, a)|}{Maps a function "\lstinline|f|" over an array "\lstinline|a|" and returns a new array.}

\descr{\lstinline|fun arrayList (a)|}{Converts an array to list (preserving the order of elements).}

\descr{\lstinline|fun listArray (l)|}{Converts a list to array (preserving the order of elements).}

\descr{\lstinline|fun foldlArray (f, acc, a)|}{Folds an array "\lstinline|a|" with a function "\lstinline|f|" and initial value "\lstinline|acc|"
  in a left-to-right manner. The function "\lstinline|f|" takes two arguments~--- an accumulator and an array element.}

\descr{\lstinline|fun foldrArray (f, acc, a)|}{Folds an array "\lstinline|a|" with a function "\lstinline|f|" and initial value "\lstinline|acc|"
  in a right-to-left manner. The function "\lstinline|f|" takes two arguments~--- an accumulator and an array element.}

\descr{\lstinline|fun iterArray (f, a)|}{Applies a function "\lstinline|f|" to each element of an array "\lstinline|a|"; does not return a value.}

\descr{\lstinline|fun iteriArray (f, a)|}{Applies a function "\lstinline|f|" to each element of an array "\lstinline|a|" and its index (index first);
  does not return a value.}

\section{Unit \texttt{Collection}}
\label{sec:collection}

Collections, implemented as AVL-trees. Four types of collections are provided: sets of ordered elements, maps of ordered keys to other values, memo
tables and hash tables. For sets and maps the generic "\lstinline|compare|" function from the unit "\lstinline|Std|" is used
as ordering relation. For memo table and hash tables the comparison of generic hash values, delivered by function "\lstinline|hash|" of unit "\lstinline|Std|"
is used. 

\subsection{Maps}

Maps are immutable structures with the following interface:

\descr{\lstinline|fun emptyMap (f)|}{Creates an empty map. An argument is a comparison function, which returns zero, positive or negative integer values depending on
the order of its arguments.}

\descr{\lstinline|fun compareOf (m)|}{Returns a comparison function, associated with the map given as an argument.}

\descr{\lstinline|fun addMap (m, k, v)|}{Adds a binding of a key "\lstinline|k|" to a value "\lstinline|v|" into a map "\lstinline|m|". As a result, a new map is
returned.}

\descr{\lstinline|fun findMap (m, k)|}{Finds a binding for a key "\lstinline|k|" in the map "\lstinline|m|". Returns the value "\lstinline|None|" if
no binding is found, and "\lstinline|Some (v)|" if "\lstinline|k|" is bound to "\lstinline|v|".}

\descr{\lstinline|fun removeMap (m, k)|}{Removes the binding for "\lstinline|k|" from the map "\lstinline|m|" and returns a new map. This function
restores a value which was previously bound to "\lstinline|k|".}

\descr{\lstinline|fun bindings (m)|}{Returns all bindings for the map "\lstinline|m|" as a list of key-value pairs, in key-ascending order.}

\descr{\lstinline|fun listMap (l)|}{Converts a list of key-value pairs into a map.}

\descr{\lstinline|fun iterMap (f, m)|}{Iterates a function "\lstinline|f|" over the bindings of map "\lstinline|m|". The function takes two
  arguments (key and value). The bindings are enumerated in an ascending order.}

\descr{\lstinline|fun mapMap (f, m)|}{Maps a function "\lstinline|f|" over all values of "\lstinline|m|" and returns a new map of results.}

\descr{\lstinline|fun foldMap (f, acc, m)|}{Folds a map "\lstinline|m|" using a function "\lstinline|f|" and initial value "\lstinline|acc|".
The function takes an accumulator and a pair key-value. The bindings are enumerated in an ascending order.}

\subsection{Sets}

Sets are immutable structures with the following interface:

\descr{\lstinline|fun emptySet (f)|}{Creates an empty set. An argument is a comparison function, which returns zero, positive or negative integer values depending on
the order of its arguments.}

\descr{\lstinline|fun compareOf (m)|}{Returns a comparison function, associated with the set given as an argument.}

\descr{\lstinline|fun addSet (s, v)|}{Adds an element "\lstinline|v|" into a set "\lstinline|s|" and returns a new set.}

\descr{\lstinline|fun memSet (s, v)|}{Tests if an element "\lstinline|v|" is contained in the set "\lstinline|s|". Returns zero if
there is no such element and non-zero otherwise.}

\descr{\lstinline|fun removeSet (s, v)|}{Removes an element "\lstinline|v|" from the set "\lstinline|s|" and returns a new set.}

\descr{\lstinline|fun elements (s)|}{Returns a list of elements of a given set in ascending order.}

\descr{\lstinline|fun union (a, b)|}{Returns a union of given sets as a new set.}

\descr{\lstinline|fun diff (a, b)|}{Returns a difference between sets "\lstinline|a|" and "\lstinline|b|" (a set of those elements
  of "\lstinline|a|" which are not in "\lstinline|b|") as a new set.}

\descr{\lstinline|fun listSet (l)|}{Converts a list into a set. }

\descr{\lstinline|fun iterSet (f, s)|}{Applied a function "\lstinline|f|" to each element of the set "\lstinline|s|". The elements are
enumerated in ascending order.}

\descr{\lstinline|fun mapSet (f, s)|}{Applies a function "\lstinline|f|" to each element of the set "\lstinline|s|" and returns a new set of images. The
elements are enumerated in an ascending order.}

\descr{\lstinline|fun foldSet (f, acc, s)|}{Folds a set "\lstinline|s|" using the function "\lstinline|f|" and initial value "\lstinline|acc|". The function
"\lstinline|f|" takes two arguments~--- an accumulator and an element of the set. The elements of set are enumerated in an ascending order.}

\subsection{Memoization Tables}

Memoization tables can be used for \emph{hash-consing}~\cite{hashConsing}~--- a data transformation which converts structurally equal
data structures into physically equal. Memoization tables are mutable; they do not work for cyclic data structures.

\descr{\lstinline|fun emptyMemo ()|}{Creates an empty memo table.}

\descr{\lstinline|fun lookupMemo (m, v)|}{Lookups a value "\lstinline|v|" in a memo table "\lstinline|m|", performing hash-consing and
  returning a hash-consed value.} 

\subsection{Hash Tables}

Hash table is an immutable map which uses hashes as keys and lists of key-value pairs as values. For hashing a generic
hash function is used, the search within the same hash class is linear with physical equality "\lstinline|==|" used for
comparison. 

\descr{\lstinline|fun emptyHashTab (n, h, c)|}{Creates an empty hash table. Argument are: a number of classes, hash and comparison functions.}

\descr{\lstinline|fun compareOf (m)|}{Returns a comparison function, associated with the hash table given as an argument.}

\descr{\lstinline|fun hashOf (m)|}{Returns a hash function, associated with the hash table given as an argument.}

\descr{\lstinline|fun addHashTab (t, k, v)|}{Adds a binding of "\lstinline|k|" to "\lstinline|v|" to the hash table "\lstinline|t|" and returns a
new hash table.}

\descr{\lstinline|fun findHashTab (t, k)|}{Searches for a binding for a key "\lstinline|k|" in the table "\lstinline|t|". Returns "\lstinline|None|"
if no binding is found and "\lstinline|Some (v)|" otherwise, where "\lstinline|v|" is a bound value.}

\descr{\lstinline|fun removeHashTab (t, k)|}{Removes a binding for the key "\lstinline|k|" from hash table "\lstinline|t|" and returns a new hash table.
  The previous binding for "\lstinline|k|" (if any) is restored.}

\section{Unit \texttt{Fun}}

The unit defines some generic functional stuff:

\descr{\lstinline|fun id (x)|}{The identify function.}

\descr{\lstinline[mathescape=false]|infixl $ after := (f, x)|}{Left-associative infix for function application.}

\descr{\lstinline|infix # after * (f, g)|}{Non-associative infix for functional composition.}

\newsavebox\factbox

\begin{lrbox}{\factbox}
\begin{lstlisting}
    fix (fun (f) { 
           fun (n) { 
             if n == 1 then 1 else n * f (n-1) fi
           }
         })     
\end{lstlisting}
\end{lrbox}

\descr{\lstinline|fun fix (f)|}{Fixpoint combinator. The argument is a knot-accepting function, thus a factorial can be
  defined as

  \usebox\factbox
}

\section{Unit \texttt{Lazy}}
\label{sec:std:lazy}

The unit provides primitives for lazy evaluation.

\descr{\lstinline|fun makeLazy (f)|}{Creates a lazy value from a function "\lstinline|f|". The function must not require any arguments.}

\descr{\lstinline|fun force (f)|}{Returns a suspended value, forcing its evaluation if needed.}

\section{Unit \texttt{List}}
\label{sec:std:list}

The unit provides some list-manipulation functions. None of the functions mutate their arguments.

\descr{\lstinline|fun singleton (x)|}{Returns a one-element list with the value "\lstinline|x|" as its head.}

\descr{\lstinline|fun size (l)|}{Returns the length of the list.}

\descr{\lstinline|fun foldl (f, acc, l)|}{Folds a list "\lstinline|l|" with a function "\lstinline|f|" and initial value "\lstinline|acc|"
  is the left-to-right manner. The function "\lstinline|f|" takes two arguments~--- an accumulator and a list element.}

\descr{\lstinline|fun foldr (f, acc, l)|}{Folds a list "\lstinline|l|" with a function "\lstinline|f|" and initial value "\lstinline|acc|"
  is the right-to-left manner. The function "\lstinline|f|" takes two arguments~--- an accumulator and a list element.}

\descr{\lstinline|fun iter (f, l)|}{Applies a function "\lstinline|f|" to the elements of the list "\lstinline|l|" in the
giver order.}

\descr{\lstinline|fun map (f, l)|}{Maps a function "\lstinline|f|" to the elements of the list "\lstinline|l|" and returns a
fresh list if images in the same order.}

\descr{\lstinline|infix +++ at + (x, y)|}{Returns the concatenation of lists "\lstinline|x|" and "\lstinline|y|".}

\descr{\lstinline|fun reverse (l)|}{Reverses a list.}

\descr{\lstinline|fun assoc (l, x)|}{Finds a value for a key "\lstinline|x|" in an associative list "\lstinline|l|". Returns
  "\lstinline|None|" if no value is found and "\lstinline|Some (v)|" otherwise, where "\lstinline|v|"~--- the first value
  whise key equals "\lstinline|x|". Uses generic comparison to compare keys.}

\descr{\lstinline|fun find (f, l)|}{Finds a value in a list "\lstinline|l|" which satisfies the predicate "\lstinline|f|". The
  predicate must return integer value, treated as boolean. Returns "\lstinline|None|" if no element satisfies "\lstinline|f|" and
  "\lstinline|Some (v)|" otherwise, where "\lstinline|v|"~--- the first value to satisfy "\lstinline|f|".
}

\descr{\lstinline|fun flatten (l)|}{Flattens an arbitrary nesting of lists into a regular list. The order of elements is preserved in both senses.}

\descr{\lstinline|fun zip (a, b)|}{Zips a pair of lists into the list of pairs. Does not work for lists of different lengths.}

\descr{\lstinline|fun unzip (a)|}{Splits a list of pairs into pairs of lists.}

\descr{\lstinline|fun remove (f, l)|}{Removes the first value, satisfying the predicate "\lstinline|f|", from the list "\lstinline|l|". The function
"\lstinline|f|" should return integers, treated as booleans.}

\descr{\lstinline|fun filter (f, l)|}{Removes all values, not satisfying the predicate "\lstinline|f|", from the list "\lstinline|l|". The function
"\lstinline|f|" should return integers, treated as booleans.}

\section{Unit \texttt{Buffer}}
\label{sec:std:buffer}

Mutable buffers.

\descr{\lstinline|fun emptyBuffer ()|}{Creates an empty buffer.}

\descr{\lstinline|fun singletonBuffer (x)|}{Creates a buffer from a single element.}

\descr{\lstinline|fun listBuffer (x)|}{Creates a buffer from a list.}

\descr{\lstinline|fun getBuffer (buf)|}{Gets the contents of a buffer as a list.}

\descr{\lstinline|fun addBuffer (buf, x)|}{Adds an element \lstinline|x| to the end of buffer \lstinline|buf| and returns the updated buffer. The buffer \lstinline|buf| can be updated in-place.}

\descr{\lstinline|fun concatBuffer (buf, x)|}{Adds buffer \lstinline|x| to the end of buffer \lstinline|buf| and returns the updated buffer. The buffer \lstinline|buf| can be updated in-place.}

\descr{\lstinline|infixl <+> before + (b1, b2)|}{Infix synonym for \lstinline|concatBuffer|.}

\descr{\lstinline|infix <+ at <+> (b, x)|}{Infix synonym for \lstinline|addBuffer|.}

\section{Unit \texttt{Matcher}}

The unit provides some primitives for matching strings against regular patterns. Matchers are immutable structures which store
string buffers with current positions. Matchers are designed to be used as stream representation for
parsers written using combinators of "\lstinline|Ostap|"; in particular, return values for "\lstinline|endOf|", "\lstinline|matchString|"
and "\lstinline|matchRegexp|" respect the conventions for such parsers.

\descr{\lstinline|fun createRegexp (r, name)|}{Creates an internal representation of regular expression; argument "\lstinline|r|" is a
  string representation of regular expression (as per function "\lstinline|regexp|"), "\lstinline|name|"~--- a string name for
diagnostic purposes.}

\descr{\lstinline|fun initMatcher (buf)|}{Takes a string argument and returns a fresh matcher.}

\descr{\lstinline|fun showMatcher (m)|}{Returns a printable representation for a matcher "\lstinline|m|" (for debugging purposes).}

\descr{\lstinline|fun endOfMatcher (m)|}{Tests if the matcher "\lstinline|m|" reached the end of string. Return value represents parsing
result as per "\lstinline|Ostap|".}

\descr{\lstinline|fun matchString (m, s)|}{Tests if a matcher "\lstinline|m|" at current position matches the string "\lstinline|s|".
Return value represents parsing result as per "\lstinline|Ostap|".}

\descr{\lstinline|fun matchRegexp (m, r)|}{Tests if a matcher "\lstinline|m|" at current position matches the regular expression "\lstinline|r|", which
  has to be constructed using the function "\lstinline|createRegexp|". Return value represents parsing result as per "\lstinline|Ostap|".}

\descr{\lstinline|fun getLine (m)|}{Gets a line number for the current position of matcher "\lstinline|m|".}

\descr{\lstinline|fun getCol (m)|}{Gets a column number for the current position of matcher "\lstinline|m|".}

\section{Unit \texttt{Ostap}}
\label{sec:ostap}

Unit "\lstinline|Ostap|" implements monadic parser combinators in continuation-passing style with memoization~\cite{MonPC,MemoParsing,Meerkat}.
A parser is a function of the shape

\begin{lstlisting}
    fun (k) {
      fun (s) {...}
    }
\end{lstlisting}

where "\lstinline|k|"~--- a \emph{continuation}, "\lstinline|s|"~--- an input stream. A parser returns either "\lstinline|Succ (v, s)|", where "\lstinline|v|"~--- some value,
representing the result of parsing, "\lstinline|s|"~--- residual input stream, or "\lstinline|Fail (err, line, col)|", where "\lstinline|err|"~--- a string, describing
a parser error, "\lstinline|line|", "\lstinline|col|"~--- line and column at which the error was encountered.

The unit describes some primitive parsers and combinators which allow to construct new parsers from existing ones.

\descr{\lstinline|fun initOstap ()|}{Clears and initializes the internal memoization tables. Called implicitly at unit initiliation time.}

\descr{\lstinline|fun memo (f)|}{Takes a parser "\lstinline|a|" and returns its memoized version. Needed for some parsers (for expamle, left-recursive ones).}

\descr{\lstinline|fun token (x)|}{Takes a string or a representation of regular expression, returned by "\lstinline|createRegexp|" (see unit \texttt{Matcher}),
  and returns a parser which recognizes exactly this string/regular expression.}

\descr{\lstinline|fun eof (k)|}{A parser which recognizes the end of stream.}

\descr{\lstinline|fun empty (k)|}{A parser which recognizes empty string.}

\descr{\lstinline|fun loc (k)|}{A parser which returns the current position (a pair "\lstinline|[line, col]|") in a stream.}

\descr{\lstinline|fun alt (a, b)|}{A parser combinator which constructs a parser alternating between "\lstinline|a|" and "\lstinline|b|".}

\descr{\lstinline|fun seq (a, b)|}{A parser combinator which construct a sequential composition of "\lstinline|a|" and "\lstinline|b|". While
  "\lstinline|a|" is a reqular parser,  "\lstinline|b|" is a \emph{function} which takes the result of parsing by "\lstinline|a|" and
returns a parser (\emph{monadicity}).}

\descr{\lstinline|infixr \| before !! (a, b)|}{Infix synonym for "\lstinline|alt|".}
    
\descr{\lstinline|infixr \|> after \| (a, b)|}{Infix synonym for "\lstinline|seq|".}

\descr{\lstinline|infix @ at * (a, f)|}{An operation which attaches a semantics action "\lstinline|f|" to a parser "\lstinline|a|". Returns a
parser which behaves exactly as "\lstinline|a|", but additionally applies "\lstinline|f|" to the result if the parsing is succesfull.}

\descr{\lstinline|fun lift (f)|}{Lifts "\lstinline|f|" into a function which ignores its argument.}

\descr{\lstinline|fun bypass (f)|}{Convert "\lstinline|f|" into a function which parser with "\lstinline|f|" but returns its argument.
  Literally, "\lstinline|bypass (f) = fun (x) {f @ lift (x)}|"}

\descr{\lstinline|fun opt (a)|}{For a parser "\lstinline|a|" returns a parser which parser either "\lstinline|a|" of empty string.}

\descr{\lstinline|fun rep0 (a)|}{For a parser "\lstinline|a|" returns a parser which parser a zero or more repetitions of "\lstinline|a|"}

\descr{\lstinline|fun rep (a)|}{For a parser "\lstinline|a|" returns a parser which parser a one or more repetitions of "\lstinline|a|"}

\descr{\lstinline|fun listBy (item, sep)|}{Constructs a parser which parses a non-empty list of "\lstinline|item|" delimited by "\lstinline|sep|".} 

\descr{\lstinline|fun list0By (item, sep)|}{Constructs a parser which parses a possibly empty list of "\lstinline|item|" delimited by "\lstinline|sep|".} 

\descr{\lstinline|fun list (item)|}{Constructs a parser which parses a non-empty list of "\lstinline|item|" delimited by ",".} 

\descr{\lstinline|fun list0 (item)|}{Constructs a parser which parses a possibly empty list of "\lstinline|item|" delimited by ",".} 

\descr{\lstinline|fun parse (p, m)|}{Parsers a matcher "\lstinline|m|" with a parser "\lstinline|p|". Returns ether "\lstinline|Succ (v)|" where
  "\lstinline|v|"~--- a parsed value, or "\lstinline|Fail (err, line, col)|", where "\lstinline|err|"~--- a stirng describing parse error, "\lstinline|line|",
  "\lstinline|col|"~--- this error's coordinates. This function may fail if detects the ambiguity of parsing.}

\descr{\lstinline|fun parseString (p, s)|}{Parsers a string "\lstinline|s|" with a parser "\lstinline|p|". Returns ether "\lstinline|Succ (v)|" where
  "\lstinline|v|"~--- a parsed value, or "\lstinline|Fail (err, line, col)|", where "\lstinline|err|"~--- a stirng describing parse error, "\lstinline|line|",
  "\lstinline|col|"~--- this error's coordinates. This function may fail if detects the ambiguity of parsing.}

\newsavebox\exprbox

\begin{lrbox}{\exprbox}
\begin{lstlisting}
    {[Left, {[token ("+"), fun (l, op, r) {Add (l, r)}],
             [token ("-"), fun (l, op, r) {Sub (l, r)}]
            }],
     [Left, {[token ("*"), fun (l, op, r) {Mul (l, r)}],
             [token ("/"), fun (l, op, r) {Div (l, r)}]
            }]}
\end{lstlisting}
\end{lrbox}

\descr{\lstinline|fun expr (ops, opnd)|}{A super-combinator to generate infix expression parsers. The argument "\lstinline|opnd|" parses primary operand, "\lstinline|ops|" is
  a list of infix operator descriptors. Each element of the list describes one \emph{precedence level} with precedence increasing from head to tail. A descriptor on
  each level is a pair, where the first element describes the associativity at the given level ("\lstinline|Left|", "\lstinline|Right|" or "\lstinline|None|") and
  the second element is a list of pairs~--- a parser for an infix operator and the semantics action (a three-argument function accepting the left parser operand, that that
  infix operator parser returns, and the right operand). For example,

  \usebox\exprbox

  specifies two levels of precedence, both left-associative, with infix operators "\lstinline|+|" and "\lstinline|-|" at the first level and
  "\lstinline|*|" and "\lstinline|/|" at the second. The semantics for these operators constructs abstract syntax trees (in this particular example the
  second argument of semantics functions is unused).
}

\section{Unit \texttt{Ref}}

The unit provides an emulation for first-class references.

\descr{\lstinline|fun ref (x)|}{Creates a mutable reference with the contents "\lstinline|x|".}

\descr{\lstinline|fun deref (x)|}{Dereferences a reference "\lstinline|x|" and returns stored value.}

\descr{\lstinline|infix ::= before := (x, y)|}{Assigns a value "\lstinline|y|" to a cell designated by the "\lstinline|x|". Returns "\lstinline|y|".}



\bibliographystyle{plainurl}
\bibliography{spec}

\end{document}
