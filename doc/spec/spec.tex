\documentclass{article}

\usepackage{amssymb, amsmath}
\usepackage{alltt}
\usepackage{pslatex}
\usepackage{epigraph}
\usepackage{verbatim}
\usepackage{latexsym}
\usepackage{array}
\usepackage{comment}
\usepackage{makeidx}
\usepackage{listings}
\usepackage{indentfirst}
\usepackage{verbatim}
\usepackage{color}
\usepackage{url}
\usepackage{xspace}
\usepackage{hyperref}
\usepackage{stmaryrd}
\usepackage{amsmath, amsthm, amssymb}
\usepackage{graphicx}
\usepackage{euscript}
\usepackage{mathtools}
\usepackage{mathrsfs}
\usepackage{multirow,bigdelim}
\usepackage{subcaption}
\usepackage{placeins}
\usepackage{xspace}

\makeatletter

\makeatother

\definecolor{shadecolor}{gray}{1.00}
\definecolor{darkgray}{gray}{0.30}

\def\transarrow{\xrightarrow}
\newcommand{\setarrow}[1]{\def\transarrow{#1}}

\def\padding{\phantom{X}}
\newcommand{\setpadding}[1]{\def\padding{#1}}

\def\subarrow{}
\newcommand{\setsubarrow}[1]{\def\subarrow{#1}}

\newcommand{\trule}[2]{\frac{#1}{#2}}
\newcommand{\crule}[3]{\frac{#1}{#2},\;{#3}}
\newcommand{\withenv}[2]{{#1}\vdash{#2}}
\newcommand{\trans}[3]{{#1}\transarrow{\padding{\textstyle #2}\padding}\subarrow{#3}}
\newcommand{\ctrans}[4]{{#1}\transarrow{\padding#2\padding}\subarrow{#3},\;{#4}}
\newcommand{\llang}[1]{\mbox{\lstinline[mathescape]|#1|}}
\newcommand{\pair}[2]{\inbr{{#1}\mid{#2}}}
\newcommand{\inbr}[1]{\left<{#1}\right>}
\newcommand{\highlight}[1]{\color{red}{#1}}
\newcommand{\ruleno}[1]{\eqno[\scriptsize\textsc{#1}]}
\newcommand{\rulename}[1]{\textsc{#1}}
\newcommand{\inmath}[1]{\mbox{$#1$}}
\newcommand{\lfp}[1]{fix_{#1}}
\newcommand{\gfp}[1]{Fix_{#1}}
\newcommand{\vsep}{\vspace{-2mm}}
\newcommand{\supp}[1]{\scriptsize{#1}}
\newcommand{\sembr}[1]{\llbracket{#1}\rrbracket}
\newcommand{\cd}[1]{\texttt{#1}}
\newcommand{\free}[1]{\boxed{#1}}
\newcommand{\binds}{\;\mapsto\;}
\newcommand{\dbi}[1]{\mbox{\bf{#1}}}
\newcommand{\sv}[1]{\mbox{\textbf{#1}}}
\newcommand{\bnd}[2]{{#1}\mkern-9mu\binds\mkern-9mu{#2}}
\newtheorem{lemma}{Lemma}
\newtheorem{theorem}{Theorem}
\newcommand{\meta}[1]{{\mathcal{#1}}}
\renewcommand{\emptyset}{\varnothing}
\newcommand{\dom}[1]{\mathtt{dom}\;{#1}}
\newcommand{\primi}[2]{\mathbf{#1}\;{#2}}
\newcommand{\sial}{S\textit{\lower -.5ex\hbox{I}\kern -.1667em\lower .5ex\hbox {A}}\kern -.125emL\@\xspace}
\definecolor{light-gray}{gray}{0.90}
\newcommand{\graybox}[1]{\colorbox{light-gray}{#1}}

\lstdefinelanguage{sial}{
keywords={fun, case, esac, do, od, if, then, else, elif, fi, skip, repeat, until, for, local},
sensitive=true,
%basicstyle=\small,
commentstyle=\scriptsize\rmfamily,
keywordstyle=\ttfamily\bfseries,
identifierstyle=\ttfamily,
basewidth={0.5em,0.5em},
columns=fixed,
fontadjust=true,
literate={->}{{$\to$}}3,
morecomment=[s]{(*}{*)}
}

\lstset{
mathescape=true,
%basicstyle=\small,
identifierstyle=\ttfamily,
keywordstyle=\bfseries,
commentstyle=\scriptsize\rmfamily,
basewidth={0.5em,0.5em},
fontadjust=true,
escapechar=!,
language=sial
}

\sloppy

\title{\sial Language Definition}

\author{Dmitry Boulytchev}

\begin{document}

\maketitle

\section{General Characteristic of the Language}

\begin{itemize}
\item procedural with first-class functions~--- functions can be passed as arguments, placed in data structures,
  returned and constructed at runtime via closures mechanism;
\item with lexical static scoping;
\item strict~--- all arguments of function application are evaluated before function's body;
\item imperative~--- variables can be re-assigned, function calls can have side effects;
\item untyped~--- no static type checking is performed;
\item supports S-expressions and pattern-matching;
\item supports user-defined infix operators, including those defined in local scopes;
\item with automatic memory management (garbage collection).
\end{itemize}

\section{Notation}

Pairs and tuples:

\[
\inbr{\bullet,\,\bullet,\,\dots}
\]

Lists of elements of kind $X$:

\[
X^*
\]

Deconstructing lists into sublists:

\[
h\circ t
\]

This applies also to lists of length 1. Empty list is denoted

\[
  \epsilon
\]


For a mapping $f : X\to Y$ we use the following definition:

\[
f [x\gets y] = \lambda\,z\,.\,
\left\{
\begin{array}{rcl}
  y    &,& x = z \\
  f\;x &,& x\neq z
\end{array}
\right.
\]

Empty mapping (undefined everywhere) is denoted $\Lambda$, the domain of a mapping $f$~--- $\dom{f}$, and we abbreviate

\[
  \Lambda[x_1\gets y_1][x_2\gets y_2]\dots[x_k\gets y_k]
\]

as

\[
  [x_1\gets y_1,\,x_2\gets y_2,\,\dots,\,x_k\gets y_k]
\]

\section{Names, Values and States}

\begin{table}[t]
  \begin{tabular}{cccl}
    denotation         & instances                       & definition                                 & comments \\
    \hline
    $\mathscr X$       & $x,\,y,\,z,\,\dots$             &                                            & variables \\
    $\mathscr T$       & $\llang{C},\,\llang{D},\,\dots$ &                                            & tags (constructors) \\
    $\Sigma$           & $\sigma$                        & $\mathscr X\to\mathscr V$                  & bindings (a partial map from variables to values) \\
    $\Sigma_{\mathscr X}$ & $\inbr{\sigma,\,S}$             & $2^{\mathscr X}\times\Sigma$                 & local scope (a set of variable and bindings) \\
    $St$               & $\inbr{\sigma_g,\,ss}$          & $\Sigma\times\Sigma^*_{\mathscr X}$           & state (global bindings and a stack of local scopes) \\
    $\mathscr L$       & $l$                             &                                            & locations \\
    $M$                & $\mu$                           & $\mathscr L\to\mathscr C$                  & abstract memory (a partial map from locations to composite values) \\
    $\mathscr V$       & $v$                             & $\mathbb Z\uplus \mathscr L$               & values (integer values or locations) \\
    $\mathscr C$       &                                 & $Arr\uplus Sexp \uplus Clo$                & composite values (arrays, S-expressions or closures) \\
    $Arr$              &                                 & $\mathbb N\times (\mathbb N\to\mathscr V)$ & arrays (length and element function) \\
    $Sexp$             &                                 & $\mathscr T \times Arr$                    & S-expressions (tag and array of subvalues) \\
    $Clo$              &                                 & $\mathscr X \times \Sigma^*_{\mathscr X}$     & closures (function name and a stack of local scopes) 
  \end{tabular}
\end{table}


\end{document}
