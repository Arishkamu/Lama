\begin{figure}[t]
  \[
    \begin{array}{rcl}
      \defterm{compilationUnit}  & : & \nonterm{import}^\star\s\nonterm{scopeExpression}\\
      \defterm{import}           & : & \term{import}\s\token{UIDENT}\s\term{;}
    \end{array}
  \]
  \caption{Compilation unit concrete syntax}
  \label{compilation_unit}
\end{figure}

\section{Compilation Units}

Compilation unit is a minimal structure recognized by a parser. An application can contain multiple units, compiled separatedly.
In order to use other units they have to be imported. In particular, the standard library is comprized of a number of precompiled units,
which can be imported by an end-user application.

The concrete syntax for compilation unit is show on Fig.~\ref{compilation_unit}. Besides optional imports a unit must contain
a \nonterm{scopeExpression}, which may contain some definitions and computations. Note, a unit can not be empty. The computations described in
a unit are performed at unit initialization time (see~\ref{separate_compilation}).

