\chapter{Driver Options and Separate Compilation}
\label{sec:driver}

Driver is a command-line unitility "\texttt{lamac}" which controls the invocation of the compiler. The
general format of invocation is

\begin{lstlisting}
    lamac $\;options\; filename$
\end{lstlisting}

Only one file name can be processed at once, the file name and the options can be specified in
arbitrary order.

The driver operates in a few modes:

\begin{itemize}
\item Interpreter mode. Performs an interpretation of a source program using the reference source-level interpreter ("\texttt{-i}") or
  compiles and runs a source on the stack machine ("\texttt{-s}"). In this mode separate compilation is not supported, thus no external
  units can be accessed (including "\lstinline|Std|"), only the standard set of builtins is available. 
\item Native mode, compilation ("\lstinline{-c}"). Compiles a source file into native code and writes an object file. All referenced
  external unit interfaces must have to be accessible; however no linking is performed and no executable is built.
\item Native mode, build (default). Same as for the native compilation, but additionally performs linking with the runtime library and
  all external units object files, generating executable.
\end{itemize}

In the native modes, the driver also creates import files ("\texttt{.i}") which are required for external units import to
work properly. These files has to reside in the same directory as object files for corresponding units.

Each natively compiled object file implicitly references all imported units; the top-level expression of each
unit is compiled into \emph{unit initialization procedure}, which calls unit initialization procedures of
all imported units in the same order these unites were imported. It is guaranteed that unit initialization
procedure for each unit will be called only once (regardless of the imports' shape for the whole application).

Additionally, the following options can be given to the driver:

\begin{itemize}
\item "\texttt{-o $filename$}"~--- specifies an alternative file name for the executable. 
\item "\texttt{-I $path$}"~--- specifies a path to look for external units. Multiples instances of this option can be given in driver's
  invocation, and the paths are looked up in that order.
\item "\texttt{-dp}"~--- forces the driver to dump the AST of compiled unit in \textsc{html} representation. The dump is written in the file with the same
  basename as the source one, with the extension replaced with "\texttt{.html}".
\item "\texttt{-ds}"~--- forces the driver to sump stack machine code. The option is only in effect in stack interpreter on
  native mode. The dump is written in the file "\texttt{.sm}".
\item "\texttt{-v}"~--- makes the driver to print the version of the compiler.
\item "\texttt{-h}"~--- makes the driver to print the help on the options.
\end{itemize}

Apart from the paths specified by the "\texttt{-I}" option the driver uses the environment variable "\texttt{LAMA}"
to locate the runtime and standard libraries (see Section~\ref{sec:stdlib}). Thus, the units from standard libraries are accessible
without any "\texttt{-I}" option given.
