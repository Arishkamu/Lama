\chapter{Standard Library}
\label{sec:stdlib}

The standard library is comprised of the runtime for the language and a set of pre-shipped units written in \lama itself.

\section{Unit \texttt{Std}}
\label{sec:std}

The unit "\lstinline|Std|" provides the interface for the runtime of the language. The implementation of
entities, defined in "\lstinline|Std|", resides in the runtime itself. The import of "\lstinline|Std|"
is added implicitly by the compiler and can not be specified by an end user.

The following declarations are accessible:

\descr{\lstinline|fun stringInt (s)|}{Converts a string representation of a signed decimal number into integer.}

\descr{\lstinline|fun read ()|}{Reads an integer value from the standard input, printing a prompt "\lstinline|>|".}

\descr{\lstinline|fun write (int)|}{Writes an integer value to the standard output.}

\descr{\lstinline|sysargs|}{A variable which holds an array of command-line arguments of the application (including the
name of the executable itself).}

\descr{\lstinline|fun makeArray (size)|}{Creates a fresh array of a given length. The elements of the array are left uninitialized.}

\descr{\lstinline|fun makeString (size)|}{Creates a fresh string of a given length. The elements of the string are left uninitialized.}

\descr{\lstinline|fun stringcat (list)|}{Takes a list of strings and returns the concatenates all its elements.}

\descr{\lstinline|fun matchSubString (subj, patt, pos)|}{Takes two strings "\lstinline|subj|" and "\lstinline|patt|" and integer position "\lstinline|pos|" and
checks if a substring of "\lstinline|subj|" starting at position "\lstinline|pos|" is equal to "\lstinline|patt|"; returns integer value, treated as a boolean.}

\descr{\lstinline|fun sprintf (fmt, ...)|}{Takes a format string (as per GNU C Library~\cite{GNUCLib}) and a variable number of arguments and
returns a string, acquired via processing these arguments according to the format string. Note: indexed arguments are not supported.}

\descr{\lstinline|fun substring (str, pos, len)|}{Takes a string, an integer position and length, and returns a substring of requested length of
  given string starting from given position. Raises an error if the original string is shorter then \lstinline|pos+len-1|.}

\descr{\lstinline|infix ++ at + (str1, str2)|}{String concatenation infix operator.}

\descr{\lstinline|fun clone (value)|}{Performs a shallow cloning of the argument value.}

\descr{\lstinline|fun hash (value)|}{Returns integer hash for the argument value; also works for cyclic data structures.}

\descr{\lstinline|fun compare (value1, value2)|}{Performs a structural deep comparison of two values. Determines a
  linear order relation for every pairs of values. Returns \lstinline|0| if the values are structurally equal, negative or
  positive integers otherwise. May not work for cyclic data structures.}

\descr{\lstinline|fun fst (value)|}{Returns the first subvalue for a given boxed value.}

\descr{\lstinline|fun snd (value)|}{Returns the second subvalue for a given boxed value.}

\descr{\lstinline|fun hd (value)|}{Returns the head of a given list.}

\descr{\lstinline|fun tl (value)|}{Return the tail of a given list.}

\descr{\lstinline|fun readLine ()|}{Reads a line from the standard input and returns it as a string. Return "\lstinline|0|" if end
of standard input was encountered.}

\descr{\lstinline|fun printf (fmt, ...)|}{Takes a format string (as per GNU C Library~\cite{GNUCLib} and a variable number of arguments and
prints these arguments on the standard output, according to the format string.}

\descr{\lstinline|fun fopen (fname, mode)|}{Opens a file of given name in a given mode. Both arguments are strings, the return value is
an external pointer to file structure.}

\descr{\lstinline|fun fclose (file)|}{Closes a file. The file argument should be that acquired by "\lstinline|fopen|" function.}

\descr{\lstinline|fun fread (fname)|}{Reads a file content and returns it as a string. The argument is a file name as a string, the file
is automatically open and closed within the call.}

\descr{\lstinline|fun fwrite (fname, contents)|}{Writes a file. The arguments are file name and the contents to write as strings. The file
is automatically created and closed within the call.}

\descr{\lstinline|fun fprintf (file, fmt, ...)|}{Same as "\lstinline|printf|", but outputs to a given file. The file argument should be that acquired
  by \lstinline|fopen| function.}

\descr{\lstinline|fun regexp (str)|}{Compiles a string representation of a regular expression (as per GNULib's regexp~\cite{GNULib}) into
  an internal representation. The return value is a external pointer to the internal representation.}

\descr{\lstinline|fun regexpMatch (pattern, subj, pos)|}{Matches a string "\lstinline{subj}", starting from the position "\lstinline|pos|",
  against a pattern "\lstinline{pattern}". The pattern is an external pointer to a compiled representation, returned by the
  function "\lstinline|regexp|". The return value is the number of matched characters.}

\descr{\lstinline|fun failure (fmt, ...)|}{Takes a format string (as per GNU C Library~\cite{GNUCLib}, and a variable number of parameters,
  prints these parameters according to the format string on the standard error and exits. Note: indexed arguments are not supported.)}

\descr{\lstinline|fun system (cmd)|}{Executes a command in a shell. The argument is a string representing a command.}

\descr{\lstinline|fun getEnv (name)|}{Returns a value for an environment variable "\lstinline|name|". The argument is a string, the
return value is either "\lstinline|0|" (if not environment variable with given name is set), or a string value.}

\section{Unit \texttt{Array}}
\label{sec:array}

Array processing functions:

\descr{\lstinline|fun initArray (n, f)|}{Takes an integer value "\lstinline|n|" and a function "\lstinline|f|" and creates an array
  \[
     \mbox{\lstinline|[f (0), f (1), ..., f (n-1)]|}
  \]
}

\descr{\lstinline|fun mapArray (f, a)|}{Maps a function "\lstinline|f|" over an array "\lstinline|a|" and returns a new array.}

\descr{\lstinline|fun arrayList (a)|}{Converts an array to list (preserving the order of elements).}

\descr{\lstinline|fun listArray (l)|}{Converts a list to array (preserving the order of elements).}

\descr{\lstinline|fun foldlArray (f, acc, a)|}{Folds an array "\lstinline|a|" with a function "\lstinline|f|" and initial value "\lstinline|acc|"
  is the left-to-right manner. The function "\lstinline|f|" takes two arguments~--- an accumulator and an array element.}

\descr{\lstinline|fun foldrArray (f, acc, a)|}{Folds an array "\lstinline|a|" with a function "\lstinline|f|" and initial value "\lstinline|acc|"
  is the right-to-left manner. The function "\lstinline|f|" takes two arguments~--- an accumulator and an array element.}

\descr{\lstinline|fun iterArray (f, a)|}{Applies a function "\lstinline|f|" to each element of an array "\lstinline|a|"; does not return a value.}

\descr{\lstinline|fun iteriArray (f, a)|}{Applies a function "\lstinline|f|" to each element of an array "\lstinline|a|" and its index (index first);
  does not return a value.}

\section{Unit \texttt{Collection}}
\label{sec:collection}

Collections, implemented as AVL-trees. Four types of collections are provided: sets of ordered elements, maps of ordered keys to other values, memo
tables and hash tables. For sets and maps the generic "\lstinline|compare|" function from the unit "\lstinline|Std|" is used
as ordering relation. For memo table and hash tables the comparison of generic hash values, delivered by function "\lstinline|hash|" of unit "\lstinline|Std|"
is used. 

\subsection{Maps}

Maps are immutable structures with the following interface:

\descr{\lstinline|fun emptyMap ()|}{Creates an empty map.}

\descr{\lstinline|fun addMap (m, k, v)|}{Adds a binding of a key "\lstinline|k|" to a value "\lstinline|v|" into a map "\lstinline|m|". As a result, a new map is
returned.}

\descr{\lstinline|fun findMap (m, k)|}{Finds a binding for a key "\lstinline|k|" in the map "\lstinline|m|". Returns the value "\lstinline|None|" if
no binding is found, and "\lstinline|Some (v)|" if "\lstinline|k|" is bound to "\lstinline|v|".}

\descr{\lstinline|fun removeMap (m, k)|}{Removes the binding for "\lstinline|k|" from the map "\lstinline|m|" and returns a new map. This function
restores a value which was previously bound to "\lstinline|k|".}

\descr{\lstinline|fun bindings (m)|}{Returns all bindings for the map "\lstinline|m|" as a list of key-value pairs, in key-ascending order.}

\descr{\lstinline|fun listMap (l)|}{Converts a list of key-value pairs into a map.}

\descr{\lstinline|fun iterMap (f, m)|}{Iterates a function "\lstinline|f|" over the bindings of map "\lstinline|m|". The function takes two
  arguments (key and value). The bindings are enumerated in an ascending order.}

\descr{\lstinline|fun mapMap (f, m)|}{Maps a function "\lstinline|f|" over all values of "\lstinline|m|" and returns a new map of results.}

\descr{\lstinline|fun foldMap (f, acc, m)|}{Folds a map "\lstinline|m|" using a function "\lstinline|f|" and initial value "\lstinline|acc|".
The function takes an accumulator and a pair key-value. The bindings are enumerated in an ascending order.}

\subsection{Sets}

Sets are immutable structures with the following interface:

\descr{\lstinline|fun emptySet ()|}{Creates an empty set.}

\descr{\lstinline|fun addSet (s, v)|}{Adds an element "\lstinline|v|" into a set "\lstinline|s|" and returns a new set.}

\descr{\lstinline|fun memSet (s, v)|}{Tests if an element "\lstinline|v|" is contained in the set "\lstinline|s|". Returns zero if
there is no such element and non-zero otherwise.}

\descr{\lstinline|fun removeSet (s, v)|}{Removes an element "\lstinline|v|" from the set "\lstinline|s|" and returns a new set.}

\descr{\lstinline|fun elements (s)|}{Returns a list of elements of a given set in ascending order.}

\descr{\lstinline|fun union (a, b)|}{Returns a union of given sets as a new set.}

\descr{\lstinline|fun diff (a, b)|}{Returns a difference between sets "\lstinline|a|" and "\lstinline|b|" (a set of those elements
  of "\lstinline|a|" which are not in "\lstinline|b|") as a new set.}

\descr{\lstinline|fun listSet (l)|}{Converts a list into a set. }

\descr{\lstinline|fun iterSet (f, s)|}{Applied a function "\lstinline|f|" to each element of the set "\lstinline|s|". The elements are
enumerated in ascending order.}

\descr{\lstinline|fun mapSet (f, s)|}{Applies a function "\lstinline|f|" to each element of the set "\lstinline|s|" and returns a new set of images. The
elements are enumerated in an ascending order.}

\descr{\lstinline|fun foldSet (f, acc, s)|}{Folds a set "\lstinline|s|" using the function "\lstinline|f|" and initial value "\lstinline|acc|". The function
"\lstinline|f|" takes two arguments~--- an accumulator and an element of the set. The elements of set are enumerated in an ascending order.}

\subsection{Memoization Tables}

Memoization tables can be used for \emph{hash-consing}~\cite{hashConsing}~--- a data transformation which converts structurally equal
data structures into physically equal. Memoization tables are mutable; they do not work for cyclic data structures.

\descr{\lstinline|fun emptyMemo ()|}{Creates an empty memo table.}

\descr{\lstinline|fun lookupMemo (m, v)|}{Lookups a value "\lstinline|v|" in a memo table "\lstinline|m|", performing hash-consing and
  returning a hash-consed value.} 

\subsection{Hash Tables}

Hash table is an immutable map which uses hashes as keys and lists of key-value pairs as values. For hashing a generic
hash function is used, the search within the same hash class is linear with physical equality "\lstinline|==|" used for
comparison. 

\descr{\lstinline|fun emptyHashTab ()|}{Creates an empty hash table.}

\descr{\lstinline|fun addHashTab (t, k, v)|}{Adds a binding of "\lstinline|k|" to "\lstinline|v|" to the hash table "\lstinline|t|" and returns a
new hash table.}

\descr{\lstinline|fun findHashTab (t, k)|}{Searches for a binding for a key "\lstinline|k|" in the table "\lstinline|t|". Returns "\lstinline|None|"
if no binding is found and "\lstinline|Some (v)|" otherwise, where "\lstinline|v|" is a bound value.}

\descr{\lstinline|fun removeHashTab (t, k)|}{Removes a binding for the key "\lstinline|k|" from hash table "\lstinline|t|" and returns a new hash table.
  The previous binding for "\lstinline|k|" (if any) is restored.}

\section{Unit \texttt{Fun}}

The unit defines some generic functional stuff:

\descr{\lstinline|fun id (x)|}{The identify function.}

\descr{\lstinline[mathescape=false]|infixl $ after := (f, x)|}{Left-associative infix for function application.}

\descr{\lstinline|infix # after * (f, g)|}{Non-associative infix for functional composition.}

\newsavebox\factbox

\begin{lrbox}{\factbox}
\begin{lstlisting}
    fix (fun (f) { 
           fun (n) { 
             if n == 1 then 1 else n * f (n-1) fi
           }
         })     
\end{lstlisting}
\end{lrbox}

\descr{\lstinline|fun fix (f)|}{Fixpoint combinator. The argument is a knot-accepting function, thus a factorial can be
  defined as

  \usebox\factbox
}

\section{Unit \texttt{Lazy}}
\label{sec:std:lazy}

The unit provides primitives for lazy evaluation.

\descr{\lstinline|fun makeLazy (f)|}{Creates a lazy value from a function "\lstinline|f|". The function must not require any arguments.}

\descr{\lstinline|fun force (f)|}{Returns a suspended value, forcing its evaluation if needed.}

\section{Unit \texttt{List}}
\label{sec:std:list}

The unit provides some list-manipulation functions. None of the functions mutate their arguments.

\descr{\lstinline|fun singleton (x)|}{Returns a one-element list with the value "\lstinline|x|" as its head.}

\descr{\lstinline|fun size (l)|}{Returns the length of the list.}

\descr{\lstinline|fun foldl (f, acc, l)|}{Folds a list "\lstinline|l|" with a function "\lstinline|f|" and initial value "\lstinline|acc|"
  is the left-to-right manner. The function "\lstinline|f|" takes two arguments~--- an accumulator and a list element.}

\descr{\lstinline|fun foldr (f, acc, l)|}{Folds a list "\lstinline|l|" with a function "\lstinline|f|" and initial value "\lstinline|acc|"
  is the right-to-left manner. The function "\lstinline|f|" takes two arguments~--- an accumulator and a list element.}

\descr{\lstinline|fun iter (f, l)|}{Applies a function "\lstinline|f|" to the elements of the list "\lstinline|l|" in the
giver order.}

\descr{\lstinline|fun map (f, l)|}{Maps a function "\lstinline|f|" to the elements of the list "\lstinline|l|" and returns a
fresh list if images in the same order.}

\descr{\lstinline|infix +++ at + (x, y)|}{Returns the concatenation of lists "\lstinline|x|" and "\lstinline|y|".}

\descr{\lstinline|fun reverse (l)|}{Reverses a list.}

\descr{\lstinline|fun assoc (l, x)|}{Finds a value for a key "\lstinline|x|" in an associative list "\lstinline|l|". Returns
  "\lstinline|None|" if no value is found and "\lstinline|Some (v)|" otherwise, where "\lstinline|v|"~--- the first value
  whise key equals "\lstinline|x|". Uses generic comparison to compare keys.}

\descr{\lstinline|fun find (f, l)|}{Finds a value in a list "\lstinline|l|" which satisfies the predicate "\lstinline|f|". The
  predicate must return integer value, treated as boolean. Returns "\lstinline|None|" if no element satisfies "\lstinline|f|" and
  "\lstinline|Some (v)|" otherwise, where "\lstinline|v|"~--- the first value to satisfy "\lstinline|f|".
}

\descr{\lstinline|fun flatten (l)|}{Flattens a list of lists into a regular list. The order of elements is preserved in both senses.}

\descr{\lstinline|fun zip (a, b)|}{Zips a pair of lists into the list of pairs. Does not work for lists of different lengths.}

\descr{\lstinline|fun unzip (a)|}{Splits a list of pairs into pairs of lists.}

\descr{\lstinline|fun remove (f, l)|}{Removes the first value, satisfying the predicate "\lstinline|f|", from the list "\lstinline|l|". The function
"\lstinline|f|" should return integers, treated as booleans.}

\descr{\lstinline|fun filter (f, l)|}{Removes all values, not satisfying the predicate "\lstinline|f|", from the list "\lstinline|l|". The function
"\lstinline|f|" should return integers, treated as booleans.}

\section{Unit \texttt{Matcher}}

The unit provides some primitives for matching strings against regular patterns. Matchers are immutable structures which store
string buffers with current positions. Matchers are designed to be used as stream representation for
parsers written using combinators of "\lstinline|Ostap|"; in particular, return values for "\lstinline|endOf|", "\lstinline|matchString|"
and "\lstinline|matchRegexp|" respect the conventions for such parsers.

\descr{\lstinline|fun createRegexp (r, name)|}{Creates an internal representation of regular expression; argument "\lstinline|r|" is a
  string representation of regular expression (as per function "\lstinline|regexp|"), "\lstinline|name|"~--- a string name for
diagnostic purposes.}

\descr{\lstinline|fun initMatcher (buf)|}{Takes a string argument and returns a fresh matcher.}

\descr{\lstinline|fun showMatcher (m)|}{Returns a printable representation for a matcher "\lstinline|m|" (for debugging purposes).}

\descr{\lstinline|fun endOfMatcher (m)|}{Tests if the matcher "\lstinline|m|" reached the end of string. Return value represents parsing
result as per "\lstinline|Ostap|".}

\descr{\lstinline|fun matchString (m, s)|}{Tests if a matcher "\lstinline|m|" at current position matches the string "\lstinline|s|".
Return value represents parsing result as per "\lstinline|Ostap|".}

\descr{\lstinline|fun matchRegexp (m, r)|}{Tests if a matcher "\lstinline|m|" at current position matches the regular expression "\lstinline|r|", which
  has to be constructed using the function "\lstinline|createRegexp|". Return value represents parsing result as per "\lstinline|Ostap|".}

\descr{\lstinline|fun getLine (m)|}{Gets a line number for the current position of matcher "\lstinline|m|".}

\descr{\lstinline|fun getCol (m)|}{Gets a column number for the current position of matcher "\lstinline|m|".}

\section{Unit \texttt{Ostap}}
\label{sec:ostap}

Unit "\lstinline|Ostap|" implements monadic parser combinators in continuation-passing style with memoization~\cite{MonPC,MemoParsing,Meerkat}.
A parser is a function of the shape

\begin{lstlisting}
    fun (k) {
      fun (s) {...}
    }
\end{lstlisting}

where "\lstinline|k|"~--- a \emph{continuation}, "\lstinline|s|"~--- an input stream. A parser returns either "\lstinline|Succ (v, s)|", where "\lstinline|v|"~--- some value,
representing the result of parsing, "\lstinline|s|"~--- residual input stream, or "\lstinline|Fail (err, line, col)|", where "\lstinline|err|"~--- a string, describing
a parser error, "\lstinline|line|", "\lstinline|col|"~--- line and column at which the error was encountered.

The unit describes some primitive parsers and combinators which allow to construct new parsers from existing ones.

\descr{\lstinline|fun initOstap ()|}{Clears and initializes the internal memoization tables. Called implicitly at unit initiliation time.}

\descr{\lstinline|fun memo (f)|}{Takes a parser "\lstinline|a|" and returns its memoized version. Needed for some parsers (for expamle, left-recursive ones).}

\descr{\lstinline|fun token (x)|}{Takes a string or a representation of regular expression, returned by "\lstinline|createRegexp|" (see unit \texttt{Matcher}),
  and returns a parser which recognizes exactly this string/regular expression.}

\descr{\lstinline|fun eof (k)|}{A parser which recognizes the end of stream.}

\descr{\lstinline|fun empty (k)|}{A parser which recognizes empty string.}

\descr{\lstinline|fun loc (k)|}{A parser which returns the current position (a pair "\lstinline|[line, col]|") in a stream.}

\descr{\lstinline|fun alt (a, b)|}{A parser combinator which constructs a parser alternating between "\lstinline|a|" and "\lstinline|b|".}

\descr{\lstinline|fun seq (a, b)|}{A parser combinator which construct a sequential composition of "\lstinline|a|" and "\lstinline|b|". While
  "\lstinline|a|" is a reqular parser,  "\lstinline|b|" is a \emph{function} which takes the result of parsing by "\lstinline|a|" and
returns a parser (\emph{monadicity}).}

\descr{\lstinline|infixr \| before !! (a, b)|}{Infix synonym for "\lstinline|alt|".}
    
\descr{\lstinline|infixr \|> after \| (a, b)|}{Infix synonym for "\lstinline|seq|".}

\descr{\lstinline|infix @ at * (a, f)|}{An operation which attaches a semantics action "\lstinline|f|" to a parser "\lstinline|a|". Returns a
parser which behaves exactly as "\lstinline|a|", but additionally applies "\lstinline|f|" to the result if the parsing is succesfull.}

\descr{\lstinline|fun lift (f)|}{Lifts "\lstinline|f|" into a function which ignores its argument.}

\descr{\lstinline|fun bypass (f)|}{Convert "\lstinline|f|" into a function which parser with "\lstinline|f|" but returns its argument.
  Literally, "\lstinline|bypass (f) = fun (x) {f @ lift (x)}|"}

\descr{\lstinline|fun opt (a)|}{For a parser "\lstinline|a|" returns a parser which parser either "\lstinline|a|" of empty string.}

\descr{\lstinline|fun rep0 (a)|}{For a parser "\lstinline|a|" returns a parser which parser a zero or more repetitions of "\lstinline|a|"}

\descr{\lstinline|fun rep (a)|}{For a parser "\lstinline|a|" returns a parser which parser a one or more repetitions of "\lstinline|a|"}

\descr{\lstinline|fun listBy (item, sep)|}{Constructs a parser which parses a non-empty list of "\lstinline|item|" delimited by "\lstinline|sep|".} 

\descr{\lstinline|fun list0By (item, sep)|}{Constructs a parser which parses a possibly empty list of "\lstinline|item|" delimited by "\lstinline|sep|".} 

\descr{\lstinline|fun list (item)|}{Constructs a parser which parses a non-empty list of "\lstinline|item|" delimited by ",".} 

\descr{\lstinline|fun list0 (item)|}{Constructs a parser which parses a possibly empty list of "\lstinline|item|" delimited by ",".} 

\descr{\lstinline|fun parse (p, m)|}{Parsers a matcher "\lstinline|m|" with a parser "\lstinline|p|". Returns ether "\lstinline|Succ (v)|" where
  "\lstinline|v|"~--- a parsed value, or "\lstinline|Fail (err, line, col)|", where "\lstinline|err|"~--- a stirng describing parse error, "\lstinline|line|",
  "\lstinline|col|"~--- this error's coordinates. This function may fail if detects the ambiguity of parsing.}

\descr{\lstinline|fun parseString (p, s)|}{Parsers a string "\lstinline|s|" with a parser "\lstinline|p|". Returns ether "\lstinline|Succ (v)|" where
  "\lstinline|v|"~--- a parsed value, or "\lstinline|Fail (err, line, col)|", where "\lstinline|err|"~--- a stirng describing parse error, "\lstinline|line|",
  "\lstinline|col|"~--- this error's coordinates. This function may fail if detects the ambiguity of parsing.}

\newsavebox\exprbox

\begin{lrbox}{\exprbox}
\begin{lstlisting}
    {[Left, {[token ("+"), fun (l, op, r) {Add (l, r)}],
             [token ("-"), fun (l, op, r) {Sub (l, r)}]
            }],
     [Left, {[token ("*"), fun (l, op, r) {Mul (l, r)}],
             [token ("/"), fun (l, op, r) {Div (l, r)}]
            }]}
\end{lstlisting}
\end{lrbox}

\descr{\lstinline|fun expr (ops, opnd)|}{A super-combinator to generate infix expression parsers. The argument "\lstinline|opnd|" parses primary operand, "\lstinline|ops|" is
  a list of infix operator descriptors. Each element of the list describes one \emph{precedence level} with precedence increasing from head to tail. A descriptor on
  each level is a pair, where the first element describes the associativity at the given level ("\lstinline|Left|", "\lstinline|Right|" or "\lstinline|None|") and
  the second element is a list of pairs~--- a parser for an infix operator and the semantics action (a three-argument function accepting the left parser operand, that that
  infix operator parser returns, and the right operand). For example,

  \usebox\exprbox

  specifies two levels of precedence, both left-associative, with infix operators "\lstinline|+|" and "\lstinline|-|" at the first level and
  "\lstinline|*|" and "\lstinline|/|" at the second. The semantics for these operators constructs abstract syntax trees (in this particular example the
  second argument of semantics functions is unused).
}

\section{Unit \texttt{Ref}}

The unit provides an emulation for first-class references.

\descr{\lstinline|fun ref (x)|}{Creates a mutable reference with the contents "\lstinline|x|".}

\descr{\lstinline|fun deref (x)|}{Dereferences a reference "\lstinline|x|" and returns stored value.}

\descr{\lstinline|infix ::= before := (x, y)|}{Assigns a value "\lstinline|y|" to a cell designated by the "\lstinline|x|". Returns "\lstinline|y|".}

