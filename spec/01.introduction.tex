\chapter{Introduction}

\lama is a programming language developed by JetBrains Research for educational purposes as an exemplary language to introduce
the domain of programming languages, compilers and tools. Its general characteristics are:

\begin{itemize}
\item procedural with first-class functions~--- functions can be passed as arguments, placed in data structures,
  returned and ``constructed'' at runtime via closure mechanism;
\item with lexical static scoping;
\item strict~--- all arguments of function application are evaluated before function body;
\item imperative~--- variables can be re-assigned, function calls can have side effects;
\item untyped~--- no static type checking is performed;
\item with S-expressions and pattern-matching;
\item with user-defined infix operators, including those defined in local scopes;
\item with automatic memory management (garbage collection).
\end{itemize}

The name \lama is an acronym for $\lambda\textsc{-Algol}$ since the language has borrowed the syntactic shape of
operators from \textsc{Algol-68}~\cite{A68}; \textsc{Haskell}~\cite{haskell} and \textsc{OCaml}~\cite{ocaml} can be
mentioned as other languages of inspiration.

The main purpose of \lama is to present a repertoire of constructs with certain runtime behavior and
relevant implementation techniques. The lack of a type system (a vital feature for a real-word language
for software engineering) is an intensional decision which allows to show the unchained diversity
of runtime behaviors, including those which a typical type system is called to prevent. On the other hand
the language can be used in future as a raw substrate to apply various ways of software verification (including
type systems) on.

The current implementation contains a native code compiler for \textsc{x86-32}, written
in \textsc{OCaml}, a runtime library with garbage-collection support, written in \textsc{C}, and a small
standard library, written in \lama itself. The native code compiler uses \textsc{gcc} as a toolchain.

In addition, a source-level reference interpreter is implemented as well as a compiler to a small
stack machine. The stack machine code can in turn be either interpreted on a stack machine interpreter, or
used as an intermediate representation by the native code compiler.

%\section{General Characteristic of the Language}

\begin{itemize}
\item procedural with first-class functions~--- functions can be passed as arguments, placed in data structures,
  returned and constructed at runtime via closures mechanism;
\item with lexical static scoping;
\item strict~--- all arguments of function application are evaluated before function's body;
\item imperative~--- variables can be re-assigned, function calls can have side effects;
\item untyped~--- no static type checking is performed;
\item supports S-expressions and pattern-matching;
\item supports user-defined infix operators, including those defined in local scopes;
\item with automatic memory management (garbage collection).
\end{itemize}

%\section{Notation}

Pairs and tuples:

\[
\inbr{\bullet,\,\bullet,\,\dots}
\]

Lists of elements of kind $X$:

\[
X^*
\]

Deconstructing lists into sublists:

\[
h\circ t
\]

This applies also to lists of length 1. Empty list is denoted

\[
  \epsilon
\]


For a mapping $f : X\to Y$ we use the following definition:

\[
f [x\gets y] = \lambda\,z\,.\,
\left\{
\begin{array}{rcl}
  y    &,& x = z \\
  f\;x &,& x\neq z
\end{array}
\right.
\]

Empty mapping (undefined everywhere) is denoted $\Lambda$, the domain of a mapping $f$~--- $\dom{f}$, and we abbreviate

\[
  \Lambda[x_1\gets y_1][x_2\gets y_2]\dots[x_k\gets y_k]
\]

as

\[
  [x_1\gets y_1,\,x_2\gets y_2,\,\dots,\,x_k\gets y_k]
\]

%\section{Names, Values and States}

\begin{table}[t]
  \begin{tabular}{cccl}
    denotation         & instances                       & definition                                 & comments \\
    \hline
    $\mathscr X$       & $x,\,y,\,z,\,\dots$             &                                            & variables \\
    $\mathscr T$       & $\llang{C},\,\llang{D},\,\dots$ &                                            & tags (constructors) \\
    $\Sigma$           & $\sigma$                        & $\mathscr X\to\mathscr V$                  & bindings (a partial map from variables to values) \\
    $\Sigma_{\mathscr X}$ & $\inbr{\sigma,\,S}$             & $2^{\mathscr X}\times\Sigma$                 & local scope (a set of variable and bindings) \\
    $St$               & $\inbr{\sigma_g,\,ss}$          & $\Sigma\times\Sigma^*_{\mathscr X}$           & state (global bindings and a stack of local scopes) \\
    $\mathscr L$       & $l$                             &                                            & locations \\
    $M$                & $\mu$                           & $\mathscr L\to\mathscr C$                  & abstract memory (a partial map from locations to composite values) \\
    $\mathscr V$       & $v$                             & $\mathbb Z\uplus \mathscr L$               & values (integer values or locations) \\
    $\mathscr C$       &                                 & $Arr\uplus Sexp \uplus Clo$                & composite values (arrays, S-expressions or closures) \\
    $Arr$              &                                 & $\mathbb N\times (\mathbb N\to\mathscr V)$ & arrays (length and element function) \\
    $Sexp$             &                                 & $\mathscr T \times Arr$                    & S-expressions (tag and array of subvalues) \\
    $Clo$              &                                 & $\mathscr X \times \Sigma^*_{\mathscr X}$     & closures (function name and a stack of local scopes) 
  \end{tabular}
\end{table}

